\subsection{Эксперименты с размерностью входного тензора}

Важным гипер-параметром от которого зависит качество распознавания и скорость обработки является размер входного изображения. Очевидно, что с ростом размера изображения качество распознавания мелких объектов улучшается, однако, вместе с тем, увеличивается как потребление памяти, так и время работы алгоритма. Для проведения эксперимента была взята модель предобученная на выборке MS-COCO (100 эпох), а затем, она была доучена на LaDD (10 эпох) и выбрана лучшая модель. Результаты эксперимента приведены в таблице ниже:

\begin{table}[H]
    \caption{Эксперименты с размерами входного изображения}\label{image-size-table}
    \begin{tabular}{|p{7cm}|p{5cm}|}
        \hline
        {Размер картинки} & {mAP (LaDD)} \\
        \hline
        1333 $\times$ 800 & 0.71 \\
        \hline
        1500 $\times$ 2000 & 0.82 \\
        \hline
        4000 $\times$ 3000 & 0.84 \\
        \hline
    \end{tabular}
\end{table}

Как видно из результатов эксперимента, модель, обученная на оригинальных картинках (4000 $\times$ 3000), практически не опережает в качестве модель, обученную на снимках, уменьшенных в два раза, однако, сильно проигрывает в скорости работы (более чем в 4 раза) и в потреблении памяти (почти в 3 раза). Модель, обученная на картинках 1333 $\times$ 800, уже сильно отстает по точности. На рисунке ниже представлены фрагменты изображений с человеком для уменьшенного и оригинального снимка:

\addimghere{6-1-image-sizes}{0.8}{Фрагменты снимков с человеком для различных размеров}{image-sizes}

С этого момента и далее, при упоминании LaDD, на вход СНС будут подаваться картинки размера 1500 $\times$ 2000 точек.