\subsection{Основные методики поиска проравших людей}

Наиболее распространенным способом поиска потерявшегося человека в природной локации является наземная поисковая операция -- это, как правило, пеший поиск с участием ответственных за подобные мероприятия служб или силами волонтёров. 

Район поиска делится на квадраты размером 500x500 метров, и прочесывается отрядом спасателей по заранее оговоренному маршруту (как правило цепью), с целью постараться увидеть или получить отклик потерявшегося человека. 

Данный способ поиска является несомненно точным, но достаточно тяжелым -- необходимо большое количество подготовленных к задаче поиска людей, необходима организация поисковых групп, необходимо выполнение многих других смежных задач. Помимо прочего пеший поиск является достаточно медленным. Поисковый квадрат размером в лесной зоне закрывается пешей поисковой группой в среднем за 3-6 часов (а таких квадратов может быть несколько).

% TODO: Insert image from SAR from film https://youtu.be/4QfOBTHEgJU 

Другим способом поиска является поиск по фотографиям с полученных с БПЛА. В последнее время появились небольшие БПЛА и их все чаще применяют спасатели для проведения поисковых работ. 

Над районом поиска составляется маршрут, по которому затем пролетает летательный аппарат, в заранее заданных точках делает фотографии с гео-меткой. После выполнения полётного задания БПЛА возвращается в поисковый штаб, где его оператор меняет на нём аккумулятор и карту памяти со снимками, после чего загружает следующее полётное задание и дрон отправляется на дальнейшие съёмки с воздуха.

Опытным путём было установлено, что наиболее оптимальная высота для осуществления аэрофотосъёмки региона поиска приблизительно 50 метров. Данная отметка выше крон деревьев при этом позволяет получить достаточно чёткие фотоснимки, на которых можно заметить человека.

% TODO: Insert images from github repo and wiki

С одной поисковой операции в среднем набирается около четырех тысяч фотографий. Полученные фотографии анализируются специалистами из ГПА (группа просмотра и анализа) с целью найти на них человека. Человек на такой фотографии может быть частично закрыт растительностью и занимает очень мало места -- в среднем взрослый лежачий человек имеет 100 пикселей в высоту и 30 в шерину при исходном размере изображения 4000x3000 пикселей и высоте сьемки 50 метров над поверхностью. Это усложняет ручной анализ изображений и пагубно сказывается на усталости анализирующего человека. В среднем опытный специалист из ГПА тратит на анализ одной фотографии около минуты, а его концентрации внимания хватает на анализ 50-70 снимков, после чего человеку необходим отдых. На анализ четырех тысяч фотографий (одна поисковая операция) у ГПА из 5-6 человек тратится около 6.5 часов.