\subsection{Основные факторы риска потерявшигося человека в прирозной зоне}

Положение потерявшегося человека часто усукубляется из-за физических и психологических факторов, таких как:

\begin{itemize}
    \item Обезвоживание. Отсутствие доступа к питьевой воде или ограниченный её запас резко снижает возможность человека поддерживать своё стабильное состояние на протяжении длительного времени, которое может потребоваться на проведение спасательной операции;
    \item Гипотермия (переохлаждение). В зимой и в ночное время температура окружающей среды снижается (особенно в природных условиях), а у потерявшигося человека как правило нет возможности согреться (если нет соответствующего снаряжения). Данная причина является одной из наиболее частых причин гибели потерявшихся людей;
    \item Травмаы. Человек, который потерялся на природе, нередко начинает паниковать, что приводит к необдуманным, импульсивным поступкам, которые нередко приводят к травмам, после получения которых шансы самостоятельно добраться до цивилизации драматически уменьшаются;
    \item Паника. Осознание потерявшимся факта невозможности самостоятельно выбраться из локации нередко приводит к панике, что резко снижает вероятность успешного выхода потерявшегося человека к цивилизации.
\end{itemize}

Эти и многие другие факторы резко снижают время выживания потерявшегося человека в лему, что в свою очередь повышает требования к временному ресурсу для поисково-спасательных операций. В среднем потерявшейся человек может продержаться в лесу в течении 6-8 дней летом и 2-4 дней зимой. Стоит также учетсь что поиск человека может начаться не сразу а спустя какое-то время. Иными словами -- часто счет идет на часы.