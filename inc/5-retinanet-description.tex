\section{Устройство RetinaNet}

Архитектура RetinaNet, взятая за основу для моего исследования, состоит из ряда основных частей. Каждая из них выполняет определенную функцию:
\begin{itemize}
    \item backbone (или базовая сеть) -- необходима для извлечения тензора признаков (feature extraction) из входного изображения. В качестве нее может быть выбран один из нескольких вариантов СНС таких как: ResNet, EfficientNet, MobileNet и прочие;
    \item Feature Pyramid Network (FPN) -- пирамидальная СНС, необходима для трансформации признаков \cite{lib-fpn}. Она объединяет карты признаков приходящих как с верхних, так и с нижних уровней базовая СНС, так как первые обладают низкой обобщающей способностью (receptive field) при большей размерности, а последние -- наоборот;
    \item Classification Subnet Network (подсеть классификации) -- необходима для извлечения информации о классах. Классификация происходит для каждого уровня пирамиды, затем результаты фильтруются;
    \item Regression Subnet Network (подсеть регрессии) -- необходима для извлечения информации о положении объекта. Регрессия также происходит для каждого уровня пирамиды, затем результаты фильтруются;
    \item Блок пост обработки -- представляет собой набор алгоритмов для фильтрации предсказаний СНС. Как правило, здесь применяются такой алгоритм как NMS (Non Maximum Suppression) \cite{lib-nms}.
\end{itemize}

Ниже приведена общая архитектурная схема RetinaNet:

\addimghere{5-retinanet-architecture}{1}{Архитектурная схема RetinaNet}{retinanet-architecture}

Рассмотрим каждую из частей RetinaNet подробнее.

\subsection{Backbone}

Как говорилось выше, базовая сеть может быть реализована по разному. Так как от ее выходных признаков зависят предсказания остальных частей детектора -- необходимо выбрать наиболее подходящую архитектуру для backbone-СНС.

Для этого были проанализированы исследования последних лет. В качестве сравнительных характеристик были выбраны accuracy и времени работы, а в качестве обучающей выборки -- ImageNet. На изображение ниже приведены результаты исследований:

\addimghere{5-1-resnet-benchmark}{1}{Сравнение различных СНС на ImageNet 2020}{retinanet-architecture}

Представленная еще в 2015 году Microsoft Research архитектура ResNet \cite{lib-resnet} оказалась настолько удачной, что и по сей день периодически ставит рекорды в области распознавания изображений. Рассмотрим семейство архитектур ResNet подробнее.

Ее успех заключается в применении Residual блоков. С ростом числа слоев в нейронной сети все острее встает проблема паралича нейронной сети -- из-за обратного распространения ошибки градиент от слоя к слою постоянно уменьшается. В результате более глубокие слои перестают обучаться и как следствие качество распознавания падает:

\addimghere{5-1-cnn-back-propagation-paralysis}{0.8}{Увеличение колличества слоев дает ходший результат}{back-propagation-paralysis}

Основная идея ResNet заключается в том, чтобы ввести так называемое "соединение с пропусками" (Skip Connection), которое пропускает один или несколько уровней, как показано на рисунке ниже:

\addimghere{5-1-residual-block}{0.6}{Residual блок}{residual-block}

Благодаря Residual блокам, градиент не уменьшается, а слои СНС можно объединять в длинные последовательности, увеличивая как обобщающую способность, так и точность такой сети:

\addimghere{5-1-resnet-training-results}{1}{Кривые обучения ResNet (справа) с 18 и 34 слоями в сравнении с аналогичной СНС без residual обоков (слева)}{residual-block}
\subsection{Feature Pyramid Network} \label{sect-5-2}

FPN состоит из нескольких частей (рис. \ref{fpn-arch}): восходящей (bottom-up) и низходящей (top-down) пирамид, а также боковых соединений (lateral connections). Рассмотрим подробнее каджую часть.

\addimghere{5-2-fpn-arch}{0.8}{Архитектура FPN}{fpn-arch}

Восходяая пирамида -- есть некоторая последовательность светросных слоев, расзмерность которых уменьшается, извлекающяя признаки из входного изображения. В нашем случае это базовая сеть (ResNet50). Стоит отметить, что по с ростом уровней пирамиды увеличивается число информации, содержащиеся в каждос супер-пикселе сверточного слоя (receptive fild), но вместе с тем уменьшается размерность выходного тензора. 
Из-за этого факта bottom-up пирамида обладает уязвимостью -- важные призноки, особенно если обьект небольной, могут потерятся при перекрытии объекта (человек прикрыт травой) а конечные обьекты имеют размер в несколтко супер-пикселей (рис. \ref{fpn-distribution}). Как следистве СНС будет работать нестабильно, а колличество ошибок 1 и 2 рода возрастет. 

\addimghere{5-2-fpn-distribution}{0.6}{Внутреннее представление человека на одном из верхних уровней FPN}{fpn-distribution}

Некоторые пути решения этой проблемы будут рассмотрены в следующих разделах, а сконцентрируемся на устройстве FPN.

Низходящая пирамида -- напротив представляет собой полследовательность всерточных слоев размерность которых увеличивается по мере спуска сигнала. На каджом шаге размерность карт признаков возрастает в 2 раза, недостающие признаки выбираются методом ближайших соседей ((рис. \ref{fpn-top-down-knn})). Начальные уровни top-down пирамиды имеют такую же размерность что и верхние уровни bottom-up пирамиды, а размеры последних -- наоборот соответствуют первым уровням backbone-сети.

\addimghere{5-2-fpn-top-down-knn}{0.5}{Увеличение размерности признаков методом ближайшего соседа}{fpn-top-down-knn}

В добавок, между пирамидами в FPN существуют и боковые соединения. Благодаря ми, признаки соответствующих слоёв поэлементно складываются, причём карты из bottom-up пирамиды проходят через свёртку $1 \times 1$ (рис. \ref{fpn-connections}).

\addimghere{5-2-fpn-connections}{0.6}{Устройство боковых соеденений}{fpn-connections}

Уровни пизходящей пирамиды принято обозначать как $P_1, P_2, ..., P_n$, притом каждый $i$ уровень top-down пирамиды соответст сответствует $i$-му уровню bottom-up пирамиды $C_1, C_2, ..., C_n$. Кллличество $P$ уровней может быть меньше от колличества уровней $С$. Тогда уровни низходящей пирамиды могут быть сдвинуты на некоторый шаг. Так оригинальная FPN имеет конфигурацию с пятью уровнями $P_3...P_7$. На изображении ниже приведен пример FPN сети с кофигурацией $P_3...P_5$:

\addimghere{5-2-fpn-p3-p5}{0.8}{Архитектура FPN c $P_3...P_5$ уровнями}{5-2-fpn-p3-p5}
\subsection{Anchor boxes}

Анкерные или якорные рамки (anchor boxes) впервые были предложены в архитектуре Faster RCNN и затем получили свое распространение на многие современные архитектуры детекторов таких как YOLO или RetinaNet. 

Предположим, что СНС сворачивает исходное изображение до тензора размерности $3 \times 3$. Таким образом, в каждом супер-пикселе выходного тензора содержится информация о некоторой области исходного изображения. Тогда можно предположить размеры объекта попадающего в такой супер-пиксель (рис. \ref{anchor-boxes}). Эти размеры, их количество и ориентация и называются анкерными рамками (т.е. рамками "привязанными" к супер-пикселю). Для RetinaNet каждый супер-пиксель имеет анкоры с соотношениями сторон $2:1, 1:1, 1:2$ и размерами $2^0, 2^{\frac{1}{3}}, 2^{\frac{2}{3}}$ (и того 9 штук). При обучении модели для каждого объекта подбираются в наиболее подходящие анкоры. Если $IoU$ рамки больше 0.5, то она считается верной, а если меньше 0.4 -- ложной. В случаях когда $IoU \in [0.4 .. 05]$ anchor box будет проигнорирован для обучения.

\addimghere{5-3-anchor-boxes}{0.6}{Аnchor boxes}{anchor-boxes}
\subsection{Подсети классификации и регрессии}

Classification и regression subnet получают на вход признаки из нисходящей пирамиды FPN и выдают предсказания о классе и местоположении объекта. 

В подсети классификации сигнал сперва проходит через 4 сверточных слоя с размерностью $[3 \times 3]$, с 256 фильтрами и со стоящей после ReLU активацией. Таким образом, в каждом convolution слое формируется тензор размера $[h \times W \times 256]$ (256 карт признаков). Затем, сигнал снова проходит через сверточный слой $[3 \times 3]$, но уже с $K \times A$ фильтрами и сигмоидальной активацией. В результате, на выходе этой сети формируется вектор длиной $K \times A$, где $K$ -- количество разных классов (в нашем случае только один класс -- это Pedestrian), $A$ -- количество анкерных рамок. Подсеть для каждого анкера выдает one-hot вектор, где позиция числа 1 соответствует номеру класса для каждого анкера к которому модель отнесла объект.

В подсети регрессии первые 4 слоя эквивалентны соответствующим слоям в classification subnet. Затем, идет свертка,  $[3 \times 3]$ формирующая $4 \times A$ карт признаков. После чего, аналогично классификационной подсети, формируется вектор длинны $4 \times A$. Таким образом, подсеть позволяет уточнить 4-компонентный вектор координат анкера под реальный размер объекта: $(\Delta x_{min}, \Delta y_{min}, \Delta x_{max}, \Delta y_{max})$.
\subsection{Функция потерь}

Отметим, что функция потерь (loss function) складывается из двух компонент: ошибки регрессии и классификации (формула \ref{eq-1}).

\begin{equation}\label{eq-1}
    L = \lambda L_{reg}+L_{cls}
\end{equation}
где:
\begin{itemize}
    \item $L$ -- функция потерь RetinaNet;
    \item $L_{reg}$ -- ошибки регрессии;
    \item $L_{cls}$ -- ошибки классификации;
    \item $\lambda$ -- коэффициент усиления, определяющий соотношение между потерями.
\end{itemize}

Рассмотрим из чего складываются потери регрессии. Мы знаем, что каждому объекту ставится в соответствии анкерная рамка. Пусть $G$ -- целевой (ground truth) объект, а $A$ -- анкер. Тогда, сопоставив объекты и anchor box-ы получим $s$ пар: $(G_i, A_i), i=1..s$.

Заметим, что для каждого анкера regression subnet отдает вектор, показывающий разницу между границами якорной рамки и объекта: $(\delta x_{min}, \delta y_{min}, \delta x_{max}, \delta y_{max})$.  Подсчитав действительное расстояние между анкером и границами объекта  $(\Delta x_{min}, \Delta y_{min}, \Delta x_{max}, \Delta y_{max})$ и сравнив его с результатами работы подсети, вычислим потери регрессии:
$$
L_{reg} = \sum_{i, j} smooth_{L1}(\delta_{ij}-\Delta_{ij})
$$
где:
\begin{itemize}
    \item $\delta_{ij}$ -- прогнозируемое расстояние между anchor box-ом и границами объекта;
    \item $\Delta_{ij}$ -- реальное расстояние между anchor box-ом и границами объекта;
    \item $i \in {x, y},\ j \in {min, max}$;
    \item $smooth_{L1}(x) = \begin{cases}\frac{1}{2}x^2, & \mid x\mid < 0\\\mid x\mid - \frac{1}{2}, & x \geq 0\end{cases}$
\end{itemize}

Для потерь классификации используется функция Focal Loss \cite{lib-focal-loss}:
$$
L_{cls} = -\sum_{i=1}^K \alpha_i \cdot log(p_i)(1-p_i)^\gamma
$$
где:
\begin{itemize}
    \item $K$ -- количество классов;
    \item $p_i$ -- вероятность с которой класса $i$ был предсказан;
    \item $\gamma$ -- фокальный коэффициент;
    \item $\alpha_i$ -- коэффициент усиления класса $i$ (в нашем случае мы имеем только один класс и $\alpha$ = 1).
\end{itemize}

Данная функция есть усовершенствование функцией кросс-энтропии \cite{lib-focal-loss}, где от модели требуется высокая степень "уверенности" и влияние часто встречающихся классов возрастает. В Focal Loss это влияние, наоборот, снижается, а наибольший вклад при обучении весов RetinaNet оказывают редко встречающиеся объекты. Делается это за счёт множителя $(1-p_i)^\gamma$, а также параметров $\alpha_1$ и $\gamma \in(0, \infty)$. Графики функций focal и cross entropy представлены на рисунке ниже:

\addimghere{5-5-focal-loss}{0.8}{Графики focal loss и cross entropy}{focal-loss}

Стоит отметить, что во время обучения, большая часть объектов, обрабатываемых классификатором, является фоном, который является отдельным классом. Поэтому может возникнуть проблема, когда нейросеть обучится определять фон лучше, чем другие объекты. Такая несбалансированность очень характерно для решаемой задачи детектирования пропавших людей. По этому, в качестве функции потерь была выбрана Focal Loss.
\subsection{Пост-обработка}

Зачастую получается так, что СНС отдает на выходе несколько прогнозов указывающих на один и тот же объект. В этом случае такие предсказания следует отфильтровать, выбрав наилучшие (рис. \ref{nms}). Однако, при фильтрации стоит учитывать случай, когда на изображении два разных объекта одного класса могут находиться рядом, и их ограничивающие рамки могут пересекаться. Эта задача решается на этапе пост-обработки с помощью алгоритма Non-Maximum Suppression (NMS).

\addimghere{5-6-nms}{0.8}{Пример работы NMS}{nms}

На вход NMS принимает набор bounding box-ов для одного класса и порог, задающий величину максимального пересечения между ними. Ограничивающие рамки сортируются по уверенности (accuracy) и кладутся на стек. В первом цикле со стека берется очередная гипотеза. Затем, во вложенном цикле со стека берется вторая гипотеза. Если между двумя гипотезами IoU больше заданного порога, то вторая гипотеза отбрасывается. Алгоритм продолжает работу до тех пор пока не переберет все пары ограничивающих рамок.

Код алгоритма может выглядеть так:

\lstinputlisting[numbers=left]{inc/scripts/nms.py}

\clearpage