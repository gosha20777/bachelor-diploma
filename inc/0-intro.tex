\anonsection{Введение}

В XXI веке роль технологий обучения и AI (англ. artificial intelligence -- искуственный интеллект) трудно переоценить. Благодоря им значительно упрощается обработка больших данных, люди могут предсказывать события, автоматизировать рпроизводственные процессы, делать новые научне открытия, и решают множество других задач и решать другие важные для человека задачи.

Одной из областей машинного обучения яаляется CV (англ. Compurer Vision -- компьютерное зение). CV позволяет обрабатывать с помощью сверточных нейронных сетей изображения или видео и на основе их анализа делать различные заключения. Благодоря этой технологии время необходимое на обработку таких данных значительно сокращается а человек избавляется от выполнения рутиной работы. Также невилируется фактор усталости человека при выполнении ручной обработки данных, что зачастую приводит к повышению качества выполнения задачи.

Отдельной областью применения является поиск пропавших или потерявшихся людей в природной зоне. В воследнее время для анализа местности все чаще применяются БПЛА (Беспилотные Летательные Аппараты) для проведения аэрофотосъемки. Процесс анализа таких изображений довольно трудоемкий для человека, по этому его можно автоматизировать с помощью технологий компьютерного зрания (CV). 

В описанном в данной работе исследовании приводится мехонизм решения описаной выше проблемы с применением современных технологий компьютерного зрения. Использование сверточных нейронных сетей сопсобно в короткие сроки решить задачу детектирования и локализации пропаышего человека на местности, что нередко может спасти ему жизнь. 

Цель работы -- исследование и разаработка различных архитектур СНС (сверточных нейронных сетей) решающих задачу детектирования потерявшихся в лесу людей по аэрофотоснимкам полученных с БПЛА, разработка пользовательского ПО (Программного Обеспечения) реализующего данную технологию с целью возможного внедрения в ПСО (поисково-спасательные отряды) для непосредственного применения. В разделе 1 описывается проблемы поиска пропаыших в лесу людей и их детектирования по аэрофотоснимкам. В резделе 2 описаны современные методы анализа изображений, принципы работы сверточных нейронных сетей и метрики их оценивания. В разделе 3 изложен процесс обора и подготовки данных для обучения нейросетевых алгоритмов. В разделе 4 производится сравнение раздичных архитектур СНС и объясняется выбор архитектуры RetinaNet-Resnet50. В разделе 5 приводится подробное описание архитектуры RetinaNet-Resnet50. В разделе 6 описан процесс обучения RetinaNet-Resnet50, исследования, улучшающие качество распознования (FPN-shift и Deep FPN), и приведены результаты экспериментов. В разделе 7 приведено пользовательского ПО. В разделе 8 приведены результаты использования данного ПО в различных ПСО.
\clearpage