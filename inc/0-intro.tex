\anonsection{Введение}

В XXI веке роль технологий обучения и AI (англ. artificial intelligence -- искусственный интеллект) трудно переоценить \cite{lib-sidorenko}. Благодаря им, значительно упрощается обработка больших данных, люди могут предсказывать события, автоматизировать производственные процессы, делать новые научные открытия, решать другие важные для человека задачи.

Одной из областей машинного обучения является CV (англ. Computer Vision -- компьютерное зрение). CV позволяет обрабатывать с помощью сверточных нейронных сетей изображения или видео и на основе их анализа делать различные заключения \cite{lib-deep-learning-book}. Благодаря этой технологии, время необходимое на обработку таких данных значительно сокращается а человек избавляется от выполнения рутиной работы. Также нивелируется фактор усталости человека при выполнении ручной обработки данных, что зачастую приводит к повышению качества выполнения задачи.

Отдельной областью применения является поиск пропавших или потерявшихся людей в природной зоне. В в последнее время для анализа местности все чаще применяются БПЛА (Беспилотные Летательные Аппараты) для проведения аэрофотосъемки. Процесс анализа таких изображений довольно трудоемкий для человека, по этому его можно автоматизировать с помощью технологий компьютерного зрения (CV). 

В описанном в данной работе исследовании приводится механизм решения описанной выше проблемы с применением современных технологий компьютерного зрения. Использование сверточных нейронных сетей способно в короткие сроки решить задачу детектирования и локализации пропавшего человека на местности, что нередко может спасти ему жизнь. 

Цель работы -- исследование и разработка различных архитектур СНС (сверточных нейронных сетей) решающих задачу детектирования потерявшихся в лесу людей по аэрофотоснимкам полученных с БПЛА, разработка пользовательского ПО (программного обеспечения) реализующего данную технологию с целью возможного внедрения в ПСО (поисково-спасательные отряды) для непосредственного применения. В разделе 1 описывается проблемы поиска пропавших в лесу людей и их детектирования по аэрофотоснимкам. В разделе 2 описаны современные методы анализа изображений, принципы работы сверточных нейронных сетей и метрики их оценивания. В разделе 3 изложен процесс сбора и подготовки данных для обучения нейросетевых алгоритмов. В разделе 4 производится сравнение различных архитектур СНС и объясняется выбор архитектуры RetinaNet-ResNet50. В разделе 5 приводится подробное описание архитектуры RetinaNet-ResNet50. В разделе 6 описан процесс обучения RetinaNet-ResNet50, исследования, улучшающие качество распознавания (сдвиг FPN, Deep FPN, Weight FPN), и приведены результаты экспериментов. В разделе 7 приведен процесс разработки пользовательского ПО. В разделе 8 приведены результаты использования данного ПО в различных ПСО.
\clearpage