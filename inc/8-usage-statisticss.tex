\section{Результаты применения ПО в поисково-спасательных отрядах}\label{sect-8}

Безусловно, главной целью рассматриваемого программного обеспечения является помощь спасателям в поисковых операциях и автоматизация процесса поиска пропавших людей. По этому, в процессе разработки ПО в приоритете всегда были удобство и простота его использования, а этап внедрения был очень важен.

Процесс внедрения ПО Lacmus начался еще на этапе его разработки и проходил постепенно и итеративно. Сперва, была подготовлена первоначальная, пробная версия ПО, имеющая минимальный базовый функционал, которая была предложена ряду добровольцев из ПСО "Лиза Алерт". Затем, от пользователей была собрана обратная связь, учтены их пожелания, исправлены ошибки в работе программы, добавлены новые функции и уже обновленная версия ПО была предложена более широкому кругу людей. Данный процесс повторяется итеративно. 

Постепенно круг пользователей ПО Lacmus начал расширяться, программа начала внедряться в другие отряды и организации, а география ее применения перестала ограничиваться Российской Федерацией. Для облегчения использования программы была создана электронная энциклопедия с методическими материалами \cite{lib-lacmus-wiki}, описывающими процесс установки на разные платформы, минимальные системные требования, графический интерфейс, требования к условиям съемки с помощью БПЛА, аспекты применения БПЛА, ответы на частозадаваемые вопросы. В добавок, был создан информационный канал с новостями проекта \cite{lib-lacmus-news} и форум технической поддержки \cite{lib-lacmus-chat}.

Все это облегчает внедрение и использование программы и способствует увеличению количества пользователей. На данный момент программное обеспечение используют такие организации и ПСО как: "Лиза Алерт", "Ангел", "Сова", "Запад", МЧС Российской Федерацией и Республики Беларусь. Все программное обеспечение имеет открытый исходный код и распространяется свободно и бесплатно \cite{lib-lacmus} \cite{lib-lacmus-app}.

В рамках внедрения ПО Lacmus некоторыми ПСО были проведены предварительные испытания его эффективности в природной среде в условиях, приближенных к реальной поисково-спасательной операции. В природной местности было размещено несколько статистов, выполняющих роль потерявшихся людей, затем, была произведена аэрофотосъемка локации с помощью БПЛА. Полученные снимки были проанализированы описанным в данном исследовании алгоритмом на месте проведения тестирования на ПЭВМ оператора БПЛА. Полученные результаты приведены в таблице:

\begin{table}[H]
    \caption{Результаты испытаний RetinaNet-WFPN различными ПСО}\label{usage-results}
    \begin{tabular}{|p{2.8cm}|p{3cm}|p{3cm}|p{3cm}|p{3cm}|}
        \hline
        {ПСО} & {Количество статистов} & {Количество найдеых статистов} & {Количество ложных срабатываний} & {Количество снимков} \\
        \hline
        Лиза Алерт (Москва) & 6 & 6 & 4 & 200 \\
        \hline
        Сова (Тула) & 5 & 5 & 2 & 100 \\
        \hline
        Ангел (Брест) & 4 & 4 & 5 & 80 \\
        \hline
    \end{tabular}
\end{table}

Как видно из таблицы, предлагаемая в данном исследовании СНС хорошо выполняет поставленную на нее задачу детектирования пропавших людей, а благодаря оптимизациям модели глубокого обучения и пользовательскому ПО -- СНС можно применять на месте проведения операции в лесу, что также ускоряет поиск. В добавок, хочется сказать, что ПСО "Ангел", Теплица Социальных технологий и другие информационные ресурсы и ПСО периодически освещают данное ПО в видео-отчетах \cite{lib-lacmus-pr1} \cite{lib-lacmus-pr2} и новостных статьях \cite{lib-lacmus-pr3} \cite{lib-lacmus-pr4} \cite{lib-lacmus-pr5}.

\clearpage