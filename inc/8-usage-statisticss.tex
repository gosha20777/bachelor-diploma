\section{Результаты применения ПО в поисково-спасательных отрядах}

Безусловно главной целью рассматреваемого программного обеспечения является помошь спасателям в поисковых операциях и автоматизация процесса поиска пропавших людей. По этому в процессе разработки ПО в приоретете всегда были удобство и простота его использования, а этап внедрения был очень важен.

Процесс внедоения ПО Lacmus начался еще на этапе его разработки и проходил постепенно и итеративно. Сперва была подготовлена первоночальная пробная версия имеющая минимальный базовый функционал, которая была предложана ряду добровольцев из ПСО "Лиза Алерт". Затем от пользователей была собрана обратная связь, учтены их пожелания исправлены ошибки в работе программы, добавлены новые функции и уже обновленная версия ПО была предложена пользователям. Данный процесс повторяется итративно. 

Постепенно круг пользователей ПО Lacmus начал расширяться, программа начала внедряться в другие отряды и организации а география ее применения перестала ограничиваться Российской Федерацией. Для облегчения использования программы была создана электронная инфеклопедия с методическими материалами, описывающими поцесс установки на разные платформы, минимальные системные требования, графическогий интрефейса, требования к условиям съемки с помошью БПЛА, аспекты применения БПЛА, ответы на частозадаваемые вопросы. В добавок был создан информационный канал с новостями проекта и форум технической поддержки.

Все это облегчает внедрение и использование программы и способстует увеличению колличества пользователей. На данный момент програмнное обеспечение используют ткие организации и ПСО как: "ЛизаАлерт", "Ангел", "Сова", "Запад", МЧС Российской Федерацией и Республики Беларусь. Все программное обеспечение имеет открытый исходный код и распространяется свободно и бесплатно.

В рамках внедрения ПО Lacmus некоторыми ПСО были проведены предварительные испытания его эффективности в природной среде в условиях приближеным к реальной поисково-спасателной операции. В природной местности было размещено несколько статистов выполняющих роль потерявшихся людей, затем была произведена аэрофотосъемка локации с помощью БПЛА. Полученные снимки были проанализированы описанным в данном исследовании алгоритмом на месте проведения тестирования на ПЭВМ оператора БПЛА. Полученные результаты приведены в таблице:

\begin{table}[H]
    \caption{Результаты испытаний RetinaNet-WFPN различными ПСО}\label{leaderboard-full}
    \begin{tabular}{|p{2.8cm}|p{3cm}|p{3cm}|p{3cm}|p{3cm}|}
        \hline
        {ПСО} & {Количество сттистов} & {Количество найдеых статистов} & {Количество ложный сработываний} & {Количество снимеов} \\
        \hline
        Лиза Алерт (Москва) & 6 & 6 & 4 & 200 \\
        \hline
        Сова (Тула) & 5 & 5 & 2 & 100 \\
        \hline
        Ангел (Брест) & 6 & 6 & 4 & 80 \\
        \hline
    \end{tabular}
\end{table}

Как видно из таблицы, предлагаемая в данном исследовании СНС хорошо выполняет поставленную на нее задачу детектирования пропавших людей, а благодоря оптимизациям модели глубокого обучения и пользовательскому ПО -- СНС можно применять на месте проведения операции в лесу, что также ускоряет поиск. В добавок хочется сказать что ПСО Ангел также предоствавил видео-отчет о проведенных испытаниях.