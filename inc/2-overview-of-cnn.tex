\section{Сверточные нейронные сети для решения основных задач компьютерного зрения}

Сверточные нейронные сети (СНС) -- открытые Яном ЛеКуном -- сейчас широко применяются в широком спектре задач анализа изображений. Успех обусловлен возможностью учета двумерной топологии изображения и устойчивости к различным искажениям изображения (изменениям масштаба и кантрастности, смещениям, поворотам и т.д.) в отличие от многослойного персептрона.

На данный момент сверточная нейронная сеть и ее модификации стремительно развиваются и считаются лучшими по точности и скорости алгоритмами нахождения объектов на сцене. Начиная с 2012 года, нейросети занимают первые места на известном международном конкурсе по распознаванию образов ImageNet, во многом обгоняя человека.
\addimghere{2-imagenet-statistics}{1}{Статистика ошибок, совершенных в процессе анализа данных ImageNet за 2010—2015}{imagenet-stat}

dd

\clearpage