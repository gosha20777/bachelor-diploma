\subsection{Сверточные нейронные сети для решения основных задач компьютерного зрения}\label{sect-2}

Сверточные нейронные сети (СНС) -- открытые Яном ЛеКуном \cite{lib-lecun-cnn} -- сейчас широко применяются в широком спектре задач анализа изображений.

Успех обусловлен возможностью учета двумерной топологии изображения и устойчивости к различным искажениям изображения (изменениям масштаба и контрастности, смещениям, поворотам и т.д.) в отличие от многослойного перцептрона \cite{lib-perciptrone}.

На данный момент сверточная нейронная сеть и ее модификации стремительно развиваются и считаются лучшими по точности и скорости алгоритмами нахождения объектов на сцене. Начиная с 2012 года, нейросети занимают первые места на известном международном конкурсе по распознаванию образов ImageNet \cite{lib-imagenet}, во многом обгоняя человека.
\addimghere{2-imagenet-statistics}{1}{Статистика ошибок, совершенных в процессе анализа данных ImageNet за 2010—2015}{imagenet-stat}

\subsection{Принцип работы СНС}

Для того, чтобы понять, как работает СНС, необходимо сначала изложить работу ее основных компонент.

\paragraph{Входной слой}

Входной слой представляет собой тензор размерности $\left[H\times W\times C\right]$, каждый элемент которого принимает нормализованное значение в диапазоне $0...1$ соответствующего пикселя входного изображения с высотой $H$, шириной $W$ и числом каналов $C$. 

Формула нормализации пикселя:

$$
f(x, \min, \max) = \frac{x-\min}{\max - \min}
$$

Где:
\begin{itemize}
    \item $f$ -- функция нормализации;
    \item $x$ -- значение цвета пикселя ($0...255$);
    \item $\min$ и $\max$ -- минимальное и максимальное значение пикселя ($0$ и $255$).
\end{itemize}
\subsubsection{Сверточный слой}

Этот слой представляет из себя набор признаков (карт) каждая из которых имеет фильтр (сканирующее ядро)\cite{lib-cnn}. Размеры всех признаков сверточного слоя -- одинаковы и определяются формулой: $(h,w) = (W_m-W_k, H_m-H_k)$. Где: $(h,w)$ -- размер карты, $W_m$ и $W_k$ -- ширина предыдущей карты и ядра, $H_m$ и $H_k$ -- их высота.

В общем виде слой слой можно описать формулой:

$$
x^i = f(x^{i-1}*k^i+b^i)
$$

Где:
\begin{itemize}
    \item $x^i$ -- выходное значение слоя $i$;
    \item $f(x)$ -- нелинейная функция активации;
    \item $k^i$ -- ядро $i$-го слоя;
    \item $b^i$ -- коэффициент сдвига слоя $i$;
    \item $*$ -- операция дискретной свертки: $(f*g)[m,n]=\sum_{k,l} f[m-k,n-l]\cdot g[k,l]$
\end{itemize}

\addimghere{2-1-2-comv-visualization}{0.5}{Операция свертки}{conv-vizialization}

Ядро представляет собой систему разделяемых весов или синапсов, это одна из главных особенностей сверточной нейросети. В многослойном перцептроне \cite{lib-perciptrone} очень много связей между нейронами, то есть синапсов, что весьма замедляет процесс детектирования. В сверточной сети -- наоборот, общие веса позволяет сократить число связей и позволить находить один и тот же признак по всей области изображения:

\addimghere{2-1-2-filter-visualization}{0.9}{Фильтр “ищет” левосторонние кривые, результат положительный}{conv-vizialization}
\paragraph{Pooling слой}

Pooling (подвыборочный) слой как и сверточный имеет признаки. Их количество совпадает с оным в предыдущем слое. Подвыборочный слой служит лишь для того чтобы уменьшить размерность признаков, отбрасывая ненужные (например усреднив их или выбрав максимальные). Слой можно описать формулой:
$$
x^i = f(a^i*subsample(x^{i-1})+b^i)
$$

Где:
\begin{itemize}
    \item $x^i$ -- выходное значение слоя $i$;
    \item $f(x)$ -- нелинейная функция активации;
    \item $a^i$, $b^i$ -- коэффициент сдвига слоя $i$;
    \item $subsample(x)$ -- операция выборки признаков (усреднение, максимизация).
\end{itemize}

Зачастую в качестве подвыборки выбирается выборка локальных максимумов (Max-Pooling):

\addimghere{2-1-3-max-pool-visualization}{0.5}{Операция подвыборки (Max Pooling)}{max-pool-visualization}

Современные сверточные нейронные сети представляют собой композицию слоев, описанных выше, а также вспомогательных подсетей, решающих ту или иную задачу компьютерного зрения (классификацию, детекцию, сегментацию и т.д.). Композиция определяет архитектуру СНС и может существенно отличаться от модели к модели. В общем виде такую композицию можно продемонстрировать так:

\addimghere{2-1-base-cnn-arch}{0.5}{Архитектура СНС решающая задачу классификации изображений}{conv-vizialization}
\subsubsection{Детектирование объектов}

Одной из основных задач компьютерного зрения является детектирование \cite{lib-detection-task}. Цель -- найти и локализовать все объекты на изображении, а также определить к какому классу относится объект. Для нахождения границ объекта для него предсказываются координаты ограничивающей рамки (англ. Bounding Box или BBox). Таким образом, задача детектирования складывается из двух задач -- регрессии и классификации. Работа детектора представлена на рисунке ниже:

\addimghere{2-2-object-detection-output}{0.8}{Пример работы детектора с классами "person" и "kite"}{object-detection-output}

\paragraph{Метрики оценивания}

Попробуем ответить на вопрос: как оценить качество работы нейронной сети? Самым очевидным будет определить долю правильных ответов (Accuracy):
$$
accuracy = \frac{P}{N}
$$
здесь $P$ -- число правильных ответов а $N$ -- размер выборки. Однако, эта метрика бесполезна в задачах с неравными классами. Пусть, имеются два класса $A$ и $B$. В выборке имеются 7 объектов класса $A$ и 3 объекта класса $B$. Теперь предположим, что обученный алгоритм будет не способен отличать классы и относить любой объект к классу $A$. Точность такого алгоритма будет равна $0.7$, что конечно неверно.

Обратимся к статистике и попробуем оценить качество алгоритма для каждого класса в отдельности. Введем такие метрики как $precision$ (точность) и $recall$ (полнота) \cite{lib-ods-metrics}:
$$
precision = \frac{TP}{TP+FP};\ \ recall = \frac{TP}{TP+FN}
$$
где:
\begin{itemize}
    \item $TP\ (true\ posetive)$ -- число истинно-положительных решений;
    \item $FP\ (false\ posetive)$ -- число ложно-положительных решений (ошибки I рода);
    \item $FN\ (false\ negative)$ -- число ложно-отрицательных решений (ошибки II рода).
\end{itemize}

С практической точки зрения, $precision$ показывает долю объектов, названных классификатором положительными и при этом, действительно являющимися положительными, а $recall$ -- какую долю объектов положительного класса из всех объектов положительного класса нашел алгоритм.

Теперь, когда мы имеем представления о $precision$ и $recall$, можем ввезти понятие average precision ($AP$) \cite{lib-map-metric}. Она определяется как площадь под precision-recall кривой (PR кривой). Кривая строится на плоскости, где ось $X$ -- это $recall$, а ось $Y$ -- $precision$. Получение множества точек «precision-recall», размерность которого совпадает с числом ответов СНС, выполняется путем выбора порогового значения оценки. При этом любое обнаружение с оценкой ниже порогового значения рассматривается как ложное срабатывание. Для каждого класса, который присутствует в результатах обнаружения, мы вычисляем $precision$ и $recall$ в этой точке. После того, как мы отобразим все пары значений «precision-recall», соответствующие каждому уникальному порогу оценки, на плоскости, мы получим PR кривую. Пример такой кривой представлен на рисунке ниже:

\addimghere{2-2-1-pr-curve}{0.6}{PR кривая}{pr-curve}

Рассмотрим еще одну метрику - $IoU$ (Intersection over Union) \cite{lib-iou-metric}. Она показывает насколько точно удалось локализовать объект и определяется отношением площади пересечения предсказанных и истинных bounding box-ов к их объединению (рис. \ref{iou-definition}). 

\addimghere{2-2-2-iou-definition}{0.4}{понятие IoU}{iou-definition}

Для оценивания качества детекторов часто применяют такие метрики как $AP@50$ или $AP@75$. Это ничто иное как метрика $AP$ при пороговом значении $0.5$ и $0.75$ $IoU$ соответственно. Если $IoU$ ниже заданного порога -- объект не учитывается. 

Напоследок, упомянем еще одну метрику -- $mAP$ (mean average precision). Как следует из названия, это просто все значения $AP$, усредненные по разным классам. С практической точки зрения, $mAP$ показывает долю объектов класса, которую нашел алгоритм, из всех объектов этого класса, однако, в добавок, учитывается уверенность модели в ложном варианте.