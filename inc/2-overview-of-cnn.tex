\section{Сверточные нейронные сети для решения основных задач компьютерного зрения}

Сверточные нейронные сети (СНС) -- открытые Яном ЛеКуном -- сейчас широко применяются в широком спектре задач анализа изображений. Успех обусловлен возможностью учета двумерной топологии изображения и устойчивости к различным искажениям изображения (изменениям масштаба и кантрастности, смещениям, поворотам и т.д.) в отличие от многослойного персептрона.

На данный момент сверточная нейронная сеть и ее модификации стремительно развиваются и считаются лучшими по точности и скорости алгоритмами нахождения объектов на сцене. Начиная с 2012 года, нейросети занимают первые места на известном международном конкурсе по распознаванию образов ImageNet, во многом обгоняя человека.
\addimghere{2-imagenet-statistics}{1}{Статистика ошибок, совершенных в процессе анализа данных ImageNet за 2010—2015}{imagenet-stat}

\subsection{Принцип работы СНС}

Для того, чтобы понять, как работает СНС, необходимо сначала изложить работу ее основных компонент.

\paragraph{Входной слой}

Входной слой представляет собой тензор размерности $\left[H\times W\times C\right]$, каждый элемент которого принимает нормализованное значение в диапазоне $0...1$ соответствующего пикселя входного изображения с высотой $H$, шириной $W$ и числом каналов $C$. 

Формула нормализации пикселя:

$$
f(x, \min, \max) = \frac{x-\min}{\max - \min}
$$

Где:
\begin{itemize}
    \item $f$ -- функция нормализации;
    \item $x$ -- значение цвета пикселя ($0...255$);
    \item $\min$ и $\max$ -- минимальное и максимальное значение пикселя ($0$ и $255$).
\end{itemize}
\subsubsection{Сверточный слой}

Этот слой представляет из себя набор признаков (карт) каждая из которых имеет фильтр (сканирующее ядро)\cite{lib-cnn}. Размеры всех признаков сверточного слоя -- одинаковы и определяются формулой: $(h,w) = (W_m-W_k, H_m-H_k)$. Где: $(h,w)$ -- размер карты, $W_m$ и $W_k$ -- ширина предыдущей карты и ядра, $H_m$ и $H_k$ -- их высота.

В общем виде слой слой можно описать формулой:

$$
x^i = f(x^{i-1}*k^i+b^i)
$$

Где:
\begin{itemize}
    \item $x^i$ -- выходное значение слоя $i$;
    \item $f(x)$ -- нелинейная функция активации;
    \item $k^i$ -- ядро $i$-го слоя;
    \item $b^i$ -- коэффициент сдвига слоя $i$;
    \item $*$ -- операция дискретной свертки: $(f*g)[m,n]=\sum_{k,l} f[m-k,n-l]\cdot g[k,l]$
\end{itemize}

\addimghere{2-1-2-comv-visualization}{0.5}{Операция свертки}{conv-vizialization}

Ядро представляет собой систему разделяемых весов или синапсов, это одна из главных особенностей сверточной нейросети. В многослойном перцептроне \cite{lib-perciptrone} очень много связей между нейронами, то есть синапсов, что весьма замедляет процесс детектирования. В сверточной сети -- наоборот, общие веса позволяет сократить число связей и позволить находить один и тот же признак по всей области изображения:

\addimghere{2-1-2-filter-visualization}{0.9}{Фильтр “ищет” левосторонние кривые, результат положительный}{conv-vizialization}
\paragraph{Pooling слой}

Pooling (подвыборочный) слой как и сверточный имеет признаки. Их количество совпадает с оным в предыдущем слое. Подвыборочный слой служит лишь для того чтобы уменьшить размерность признаков, отбрасывая ненужные (например усреднив их или выбрав максимальные). Слой можно описать формулой:
$$
x^i = f(a^i*subsample(x^{i-1})+b^i)
$$

Где:
\begin{itemize}
    \item $x^i$ -- выходное значение слоя $i$;
    \item $f(x)$ -- нелинейная функция активации;
    \item $a^i$, $b^i$ -- коэффициент сдвига слоя $i$;
    \item $subsample(x)$ -- операция выборки признаков (усреднение, максимизация).
\end{itemize}

Зачастую в качестве подвыборки выбирается выборка локальных максимумов (Max-Pooling):

\addimghere{2-1-3-max-pool-visualization}{0.5}{Операция подвыборки (Max Pooling)}{max-pool-visualization}

Современные сверточные нейронные сети представляют собой композицию слоев, описанных выше, а также вспомогательных подсетей, решающих ту или иную задачу компьютерного зрения (классификацию, детекцию, сегментацию и т.д.). Композиция определяет архитектуру СНС и может существенно отличаться от модели к модели. В общем виде такую композицию можно продемонстрировать так:

\addimghere{2-1-base-cnn-arch}{0.5}{Архитектура СНС решающая задачу классификации изображений}{conv-vizialization}

\clearpage