\subsection{Пользовательское программное обеспечение}\label{sect-7}

Запуск СНС, как правило, требует от пользователя определенных навыков и умений, что сужает возможности использование СНС в ПСО. Для повышения удобства использования разработанного алгоритма машинного обучения (раздел \ref{sect-wfpn}), было принято решение разработать пользовательское программное обеспечение (ПО). В процессе разработки ПО учитывались следующие принципы:
\begin{itemize}
    \item простота -- ПО должно быть простым в использовании и установке;
    \item платформонезависимость -- ПО должно работать на всех современных персональных компьютерах и операционных системах (linux, windows, osx);
    \item низкое потребление ресурсов -- ПО должно быть эффективным по памяти и по времени;
    \item простота разработки -- ПО должно обеспечивать высокую платформонезависимость программного кода для снижения стоимости разработки и поддержки.
\end{itemize}

Ниже будут описаны основные аспекты процесса разработки пользовательского ПО и принципы построения его архитектуры, а также причины выбора тех-или иных технологий для разработки.

\subsection{Оптимизация модели глубокого обучения} \label{sect-7-1}

Для возможности использования разработанной архитектуры WFPN на ПЭВМ и практического применения СНС в ПСО необходимо было как снизить ее требования к ресурсам, так и сократить время работы алгоритма или инференс (англ. inference) на портативных устройствах.

Для осуществления этих целей прибегают к следующим подходам:
\begin{itemize}
    \item Кванитизация -- это процесс уменьшения размера СНС в оперативной памяти компьютера. Уменьшение потребляемой памяти осуществляется путем уменьшения количества информации, необходимое для хранения одного параметра (веса) СНС. Так, например, изначально параметры модели имеют тип данных FP32. В процессе кванитизации тип данных может измениться на FP16 или даже INT8. Помимо снижения потребляемой памяти, кванитизация может привезти к уменьшению времени работы на различных устройствах. К примеру, на графических ускорителях nVidia модели с весами в формате FP16 могут выполняться до 2 раз быстрее, а кванитизация в INT8 ускоряет выполнение нейронных сетей на ARM устройствах;
    \item Объединение слоев СНС. Зачастую, некоторые слои СНС могут быть объеденины (например, свертка и ReLU активация), что приводит к увеличению производительности;
    \item Использование NPU (Neural Processor Unit) и TPU (Tensor Processor Unit) -- сопроцессоров, ускоряющих работу нейронных сетей;
    \item Использование аппаратно-ориентированных низкоуровневых библиотек, задействующих аппаратные возможности той, или иной платформы (например, AVX инструкции) для ускорения времени инференса.
\end{itemize}

Для практической реализации вышеупомянутых подходов мною были использованы следующие инструменты, библиотеки и устройства:

\begin{itemize}
    \item OneDNN (Tensorflow-oneDNN-2.3) - низкоуровневая библиотека обеспечивающая высокопроизводительное выполнение тензорных операций на x86 процессорах Intel и AMD \cite{lib-onednn};
    \item CUDA \cite{lib-cuda}, CuDNN \cite{lib-cudnn}, TensorRT \cite{lib-tensorrt} (Tensorflow-tensorrt-2.3) -- набор низкоуровневых библиотек и инструментов обиспечиваюие высокопроизводительное выполнение тензорных операций на графических ускорителях nVidia, а также объединение слоев и кванитизацию;
    \item Сопроцессоры Intel Movidius Myriad 2 \cite{lib-movidius} и Google Coral Edge TPU \cite{lib-coral} (рис. \ref{7-1-npus}) и соответствующие им наборы библиотек (OpenVINO \cite{lib-openvino} и Tensorflow-lite \cite{lib-tflite}).
\end{itemize}

Полученные результаты ускорения модели RetinaNet-WFPN приведены в таблице ниже:

\begin{table}[H]
    \caption{Оптимизация времени инференса RetinaNet-WFPN}\label{leaderboard-full}
    \begin{tabular}{|p{6cm}|c|c|p{2cm}|}
        \hline
        {Устройство} & {Библиотека} & {Формат} & {Время работы, мс} \\
        \hline
        Intel i7-9750H (6/12) @ 4.500GHz & Tensorflow-cpu-2.3 & FP32 & 1100 \\
        \hline
        Intel i7-9750H (6/12) @ 4.500GHz & Tensorflow-oneDNN-2.3 & FP16 & 800 \\
        \hline
        nVidia Quadro T1000 Mobile (4 Gb) & Tensorflow-gpu-2.3 & FP32 & 400 \\
        \hline
        nVidia Quadro T1000 Mobile (4 Gb) & Tensorflow-tensorrt-2.3 & FP16 & 300 \\
        \hline
        Intel i7-9750H (6/12) @ 4.500GHz + Intel Movidius Myriad 2 & OpenVINO & FP16 & 500 \\
        \hline
        Intel i7-9750H (6/12) @ 4.500GHz + Google Coral Edge TPU & Tensorflow-lite & INT8 & 240 \\
        \hline
    \end{tabular}
\end{table}

Из полученных результатов видно, что благодаря оптимизациям удалось уменьшить время работы алгоритма для CPU в 1.4 раза, для GPU в 1.3 раза. С применением сопроцессоров Intel Movidius Myriad 2 и Google Coral Edge TPU этот показатель составил 2.2 и 4.5 раза соответственно. В связи с этим, в дальнейшем для запуска СНС на ПЭВМ будем использовать библиотеки Tensorflow-oneDN для CPU, Tensorflow-tensorrt для GPU, а также, OpenVINO и Tensorflow-lite для сопроцессоров.

\addimghere{7-1-npus}{0.8}{Сопроцессоры Google Coral Edge TPU (сверху) и Intel Movidius Myriad 2 (снизу)}{7-1-npus}
\subsection{Архитектура пользовательского ПО}
Принимая во внимание, что пользовательское ПО должно работать на разных платформах (Windows, Linux, OSX), иметь единый, платформонезависимый код и при этом обладать низким потреблением ремурсов мною выби выбраны следующие технологии для разработки: язык программирования C\# и платформа dot net core, библиотеки ReactiveUI и AvaloniaUI для построения ПО с графичиским интерфейсом (GUI, Graphic User Interface). Опишем каждый из компонентов более подробно:

\begin{itemize}
    \item .NET core -- это кроссплатформенная управляемая программная среда с открытым исходным кодом позваляющая разработывать ПО для операционных систем Windows, Linux и macOS для архитектур ARM и x66;
    \item ReactiveUI -- набор библиотек для .NET, позволяющий разработывать приложения с приминением реактивной модели программирования и паттерна MVVM (Model-Viev-View Мodel). Данный подход позволяет разделить ПО на независимые модули: Viev (инкапсулирует в мебе представление ПО -- графический интерфейс), Model (содержит в себе бизнес-логику приложения), View Мodel (связывает Viev и Model предостваляя интерфейсу набор команд и привязок);
    \item AvaloniaUI -- набор библиотек для .NET позволяющий проектировать и отрисовывать GUI любой сложности для различных платформ. По сравнению с аналогами (Electron, GTK, QT, WPF) он обладает более низким потреблением ресурсов и высокой степенью интеграции с .NET core.
\end{itemize}

Пример интерфейса пользовательского ПО приведен на рисунке ниже:

\addimghere{7-2-gui}{0.8}{Главное окно программы}{7-2-gui}

\subsubsection{Запуск моделей глубокого обучения в .NET}

Как говорилось в разделе \ref{sect-7-1}, для высокопроизводительного инференса моделей глубокого обучения требуются низкоуровневые библиотеки. Эти библиотеки спроектированы для разного оборудования и имеют различные интерфейсы взаимодействия (API). Также, зачастую, для возможности использования того или иного оборудования необходима установка различных драйверов (например nVidia CUDA и CuDNN). В добавок, использование нейронных сетей часто требует наличие интерпретатора python и различных python зависимостей (например opencv-python, numpy и т д). Все это требует от пользователя дополнительных знаний и затрудняет установку ПО и использование СНС.

Для решения этой проблемы была разработана система модулей \cite{lib-plugins}. Основная идея заключается в том, что модель поставляется вместе с модулем, включающем в себя весь набор низкоуровневых библиотек, необходимых для работы того, или оного оборудования, а также программный код осуществляющий связку между С\slash С++ методами библиотек и .NET CLI (Common Language Infrastructure). Таким образом программный код C\# способен вызывать соответствующие низкоуровневые методы и осуществлять вычисления на том, или ином устройстве.

В свою очередь, каждый модуль предоставляет единообразный интерфейс взаимодействия -- $IObjectDetectionPlugin$. С помощью него, программа может узнать, на каких операционных системах способен работать модуль и с какими устройствами. Модули могут подключаться в программу независимо друг от друга. ПО анализирует платформу, на которой оно запущено, и предлагает установить пользователю модуль, обеспечивающий максимальную производительности на его оборудовании. Например, если у пользователя имеется GPU от nVidia -- программа предложит пользователю использовать модуль с CUDA, CuDNN и TensorRT. Все модули хранятся на удаленном сервере, а пользовательское ПО способно управлять ими по средствам менеджера модулей (система управления модулей схожа с менеджерами пакетов APT, PIP, NuGet). Исходный код интерфейсов приведен в приложении А.

Классификация платформ запуска моделей глубокого обучения представлена ниже:

\begin{itemize}
    \item Операционные системы:
    \begin{itemize}
        \item Linux
        \item Windows
        \item OSX
    \end{itemize}
    \item Вычислительные устройства:
    \begin{itemize}
        \item CPU
        \item GPU
        \begin{itemize}
            \item nVidia GPU
            \item AMD GPU
        \end{itemize}
        \item Сопроцессоры
        \begin{itemize}
            \item Google Edge TPU
            \item Intel Movidius NPU
        \end{itemize}
    \end{itemize}
\end{itemize}

Для работы с различными вычислительными устройствами используются различные наборы библиотек (так, например, для процессоров Intel и AMD используется oneDNN, для GPU от nVidia -- CUDA, CuDNN и TensorRT, а для GPU от AMD -- ROCm и DirectML). Применение той, или иной библиотеки может зависеть от операционной системы. С учетом этих особенностей мною были использованы следующие варианты конфигураций:

\addimghere{7-2-plotforms-tree}{0.8}{Дерево платформ и библиотек}{7-2-plotforms-tree}

Собранный модуль имеет следующую структуру: в корне модуля лежат .NET Core библиотеки, обеспечивающие вызов низкоуровневых компонентов и предоставляющие программе API для инициализации и вызова модуля, а в каталоге runtimes, находятся платформозависимые низкоуровневые компоненты (рис. \ref{7-2-plugin-tree}).

\addimghere{7-2-plugin-tree}{0.8}{Cтруктура модуля LаcmusRetineNetPlugin.CPU}{7-2-plugin-tree}

Благодаря предложенному подходу, пользователь может легко манипулировать модулями с моделями глубокого обучения в зависимости от его потребностей. Инкапсуляция в модулях низкоуровневых библиотек избавляет пользователя от ручной установки драйверов и компонентов, облегчая использование ПО. В добавок, такая система позволяет обновлять СНС без необходимости обновления самого ПО и его рекомпиляции, что упрощает процесс разработки и поддержки.

Общая схема архитектуры ПО приведена ниже:

\addimghere{7-2-app-arch}{0.8}{J,ofz fh[LacmusApp]}{7-2-app-arch}

Напоследок, хочется отметить, что данное ПО и все его компоненты (включая СНС) имеют открытый исходный код и распространяются свободно под лицензией GNU GPL v3 \cite{lib-lacmus} \cite{lib-lacmus-app}.
