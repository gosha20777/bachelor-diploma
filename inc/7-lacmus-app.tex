\section{Пользовательское программное обеспечние}

Запуск СНС как правило требует от пользователя поределенных навыков и умений, что сужает возможнсти сипользование СНС в ПСО. Для повышения удобства использования разработанного алгоритма машинного обучения было принято решение разработать пользовательское программное обеспечение (ПО). В процессе разработки ПО учитывались следубщие принципы:
\begin{itemize}
    \item простота -- приложение должно быть простым в использовании и утановке;
    \item платформонезависимость -- приложение должно работать на всех современных персональных компьютерах и операционных системах (linux, windows, osx);
    \item низкое потребление ресурсов -- прилодение должно быть эффетивным по памяти и по времмени;
    \item простота разработки -- прилодение должно обеспечивать высокую платформонезависимость программного кода для снижения стоимости разработки и поддержки.
\end{itemize}

Ниже будут описаны основные спекты процесса разработки прользовательского ПО и принципы построения его архитектуры а также причины выбора тех-или иных технологий для разработки.

\subsection{Оптимизиция модели глубокого обучения} \label{sect-7-1}

Для возможности использования разработанной архитектуры WFPN на ПЭВМ и практического применения СНС в ПСО необходимо было как снизить ее требования к ресурсам, так и сократить пермя работы алгоритма или инференс (англ. inference) на портативнфх устройствах.

Для осущиствления этих целей прибегают к следующим подхожам:
\begin{itemize}
    \item кванитизация -- это процесс уменьшения размера СНС в оперативной памяти компьютера. Умешьшение потребляемой памяти оссуществляется путем уменьшения колличества информации необходимое для хранения одного параметра (веса) СНС. Так например, изначально параметры модели имеют тип данных FP32. В процессе кванитизации тип данных может измениться на FP16 или даже INT8. Помимо снижения потребляемой памяти кванитизация может привезти к уменьшению времени работы на различных устройствах. На графическиих ускорителях nVidia модели с весами в формате FP16 могут выполняться до 2 раз быстрее, а кванитизаций в INT8 ускоряет выполнение нейронных сетей на ARM устройствах;
    \item Объединение слоев СНС -- зачастую некоторые слои СНС могут быть объеденены (наппример свертка и ReLU активация), что приводит к увеличению производительности;
    \item Использование NPU (Neural Proceccor Unit) и TPU (Tensor Proceccor Unit) - сопроцессоров, ускоряющих работу нейронных сетей;
    \item Использование аппаратно-ориентированных низкоуровнивых библиотек, задействующих аппаратные возможности той или оний платформы (например AVX инструкции) для ускорения времени инференса.
\end{itemize}

Для практической реализации вышеупомянутых подходов мною были использованы следубщие инструменты, библиотеки и устройства:

\begin{itemize}
    \item OneDNN (Tensorflow-oneDNN-2.3) - низкоуровнивая библиотека обиспечивающая высокопроизводительное выполнение тензорных операций на x86 процессорах Intel и AMD;
    \item Cuda, CuDNN, TensorRT (Tensorflow-tensorrt-2.3) -- набор низкоуровневых библиотек и инструментов обиспечиваюие высокопроизводительное выполнение тензорных операций на графических ускорителях nVidia, а также объединение слоев и кванитизацию;
    \item Сопроцессоры Intel Movidius Myriad 2 и Google Coral Edge TPU (рис. \ref{7-1-npus}) и соответвтубщие им наборы библиотек (OpenVINO и Tensorflow-lite).
\end{itemize}

Полученные результаты ускорения модели RetinaNet-WFPN приведены в таблице ниже:

\begin{table}[H]
    \caption{Оптимизация времени инференса RetinaNet-WFPN}\label{leaderboard-full}
    \begin{tabular}{|p{6cm}|c|c|p{2cm}|}
        \hline
        {Устройство} & {Библиотека} & {Формат} & {Время работы, мс} \\
        \hline
        Intel i7-9750H (6/12) @ 4.500GHz & Tensorflow-cpu-2.3 & FP32 & 1100 \\
        \hline
        Intel i7-9750H (6/12) @ 4.500GHz & Tensorflow-oneDNN-2.3 & FP16 & 800 \\
        \hline
        nVidia Quadro T1000 Mobile (4 Gb) & Tensorflow-gpu-2.3 & FP32 & 400 \\
        \hline
        nVidia Quadro T1000 Mobile (4 Gb) & Tensorflow-tensorrt-2.3 & FP16 & 300 \\
        \hline
        Intel i7-9750H (6/12) @ 4.500GHz + Intel Movidius Myriad 2 & OpenVINO & FP16 & 500 \\
        \hline
        Intel i7-9750H (6/12) @ 4.500GHz + Google Coral Edge TPU & Tensorflow-lite & INT8 & 240 \\
        \hline
    \end{tabular}
\end{table}

Из полученных результатов видно что благодаря оптимизациям удалось уменьшить время работы алгоритма для CPU в 1.4 раза, для GPU в 1.3 раза. С применением сопроцессоров Intel Movidius Myriad 2 и Google Coral Edge TPU этот показатель составил 2.2 и 4.5 раза соответсвенно. В связи с этим в дальнейшем для запуска СНС на ПЭВМ будем использовать библиотеки Tensorflow-oneDN для CPU, Tensorflow-tensorrt для GPU, OpenVINO и Tensorflow-lite для сопроцессоров.

\addimghere{7-1-npus}{0.8}{Сопроцессоры Google Coral Edge TPU (сверху) и Intel Movidius Myriad 2 (снизу)}{7-1-npus}
\subsection{Архитектура пользовательского ПО}
Принимая во внимание, что пользовательское ПО должно работать на разных платформах (Windows, Linux, OSX), иметь единый, платформонезависимый код и при этом обладать низким потреблением ремурсов мною выби выбраны следующие технологии для разработки: язык программирования C\# и платформа dot net core, библиотеки ReactiveUI и AvaloniaUI для построения ПО с графичиским интерфейсом (GUI, Graphic User Interface). Опишем каждый из компонентов более подробно:

\begin{itemize}
    \item .NET core -- это кроссплатформенная управляемая программная среда с открытым исходным кодом позваляющая разработывать ПО для операционных систем Windows, Linux и macOS для архитектур ARM и x66;
    \item ReactiveUI -- набор библиотек для .NET, позволяющий разработывать приложения с приминением реактивной модели программирования и паттерна MVVM (Model-Viev-View Мodel). Данный подход позволяет разделить ПО на независимые модули: Viev (инкапсулирует в мебе представление ПО -- графический интерфейс), Model (содержит в себе бизнес-логику приложения), View Мodel (связывает Viev и Model предостваляя интерфейсу набор команд и привязок);
    \item AvaloniaUI -- набор библиотек для .NET позволяющий проектировать и отрисовывать GUI любой сложности для различных платформ. По сравнению с аналогами (Electron, GTK, QT, WPF) он обладает более низким потреблением ресурсов и высокой степенью интеграции с .NET core.
\end{itemize}

Пример интерфейса пользовательского ПО приведен на рисунке ниже:

\addimghere{7-2-gui}{0.8}{Главное окно программы}{7-2-gui}

\subsubsection{Запуск моделей глубокого обучения в .NET}

Как говорилось в разделе \ref{sect-7-1} для высокопроизводительного инференса моделей глубокого обучения требуются низкоуровневые библиотеки. Эти библиотеки спроектированны для разного оборудования и имеют различные интерфейсы взаимодействия (API). Также зачастую для возможности использования того или иного оборудования необходима установка различных драйверов (например nVidia CUDA и CuDNN). В добавок использование нейронных сетей часто требует наличие интерпритатора python и различных python замисимостей (например opencv-python, numpy и т д). Все это требует от пользователя дополнительныъ знаний и затрудняет установку ПО и использование СНС.

Для решения этой проблемы была разработана система модулей. Основная идея заключается в том что модель поставляется вместе с модулем, включающем в себя весь набор необходимых низкоруровневых библиотек, необходимых для работы того или оного оборудования а также программный код оссушествляющий связку между С\slash С++ методами библиотек и .NET CLI (Common Language Infrastructure). Таким образом программный код C\# способен вызывать соответствующие низкоуровневые методы и осуществлять вычисления на том или ином устройстве.

В свою очередь каждый модуль предоставляет единообразный интрефейс взаимодействия -- IObjectDetectionPlugin. С помошью него программа может узнать на каких операционных системах способен работать модуль и с какими устройствами. Модули могут подключаться в программу независимо друг от друга. ПО анализирует платформу на которой оно запущено и предлагает установить пользователю модуль, обеспечивабщий максимальную производительности на его оборудовании. Например если у пользователя имеется GPU от nVidia -- программа предложит пользователю использовать модуль с CUDA, CuDNN и TensorRT. Все модули хранятся на удаленном сервере, а пользовательское ПО способно управлять ими по средствам менеджера модулей (система управления модулей схожа с менеджерами пакетов APT, PIP, NuGet).

Классификация платформ запуска моделей глубокого обучения представлены ниже:

\begin{itemize}
    \item Операционные системы:
    \begin{itemize}
        \item Linux
        \item Windows
        \item OSX
    \end{itemize}
    \item Вычислительные устройства:
    \begin{itemize}
        \item CPU
        \item GPU
        \begin{itemize}
            \item nVidia GPU
            \item AMD GPU
        \end{itemize}
        \item Сопроцессоры
        \begin{itemize}
            \item Google Edge TPU
            \item Intel Movidius NPU
        \end{itemize}
    \end{itemize}
\end{itemize}

Для работы с различными вычислительными устройствами используются различные наборы библиотек (так например для процессоров Intel и AMD споользуется oneDNN, для GPU от nVidia -- CUDA, CuDNN и TensorRT а для GPU от AMD -- ROCm и DirectML). Применение той или иной библиотеки модет зависеть от операционной системы. С учетом этих особенностий мною были использованы следубщие варианты конфигураций:

\addimghere{7-2-plotforms-tree}{0.8}{Дерево платформ и библиотек}{7-2-plotforms-tree}

Собранный модуль меет следующую структуру: в корне модуля лежат .DET Core библиотеки обеспечивающие вызов низкоуровневых компонентов и предоставляющие программе API для инициализации и вызова модуля, а в каталоге runtimes находятся платформозависимые низкоуровневые компоненты (рис. \ref{7-2-plugin-tree}).

\addimghere{7-2-plugin-tree}{0.8}{Cтруктура модуля LаcmusRetineNetPlugin.CPU}{7-2-plugin-tree}

Благодаря предложенному подходу пользователь может легко манипулировать модулями с моделями грубокого обучения в зависимости от его потребностей. Инкапсуляция в модулях низкоуровневых библиотек избавляет пользователя от ручной установки драйверов и компонентов, облегчая использование ПО. В добавок такая система позволяет обновлять СНС без необходимости обновления самого ПО и его рекомпиляции, что упрощает процесс разработки и поддержки.

Общая схема архитектуры ПО приведена ниже:

\addimghere{7-2-app-arch}{0.8}{J,ofz fh[LacmusApp]}{7-2-app-arch}

На последок хочется отметить что данное ПО и все его компоненты (включая СНС) имеют открытый исходный код и распространяются свободно под лицензией GNU GPL v3.


\clearpage