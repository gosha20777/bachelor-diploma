\subsubsection{Pooling слой}

Pooling (подвыборочный) слой как и сверточный имеет признаки. Их количество совпадает с оным в предыдущем слое. Подвыборочный слой служит лишь для того чтобы уменьшить размерность признаков, отбрасывая ненужные (например усреднив их или выбрав максимальные). Слой можно описать формулой:
$$
x^i = f(a^i*subsample(x^{i-1})+b^i)
$$

Где:
\begin{itemize}
    \item $x^i$ -- выходное значение слоя $i$;
    \item $f(x)$ -- нелинейная функция активации;
    \item $a^i$, $b^i$ -- коэффициент сдвига слоя $i$;
    \item $subsample(x)$ -- операция выборки признаков (усреднение, максимизация).
\end{itemize}

Зачастую в качестве подвыборки выбирается выборка локальных максимумов (Max-Pooling):

\addimghere{2-1-3-max-pool-visualization}{0.5}{Операция подвыборки (Max Pooling)}{max-pool-visualization}