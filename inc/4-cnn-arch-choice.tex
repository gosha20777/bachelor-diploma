\section{Архитектуры нейросетевых детекторов}\label{sect-4}

Различают несколько типов нейросетевых детекторов: One-stage и two-stage \cite{lib-detector-types}.

Two-stage -- более старый подход детекции. Он состоит из двух этапов. Сначала генерируются области, которые соответствуют искомым объектам. Ненужные области отсеиваются с помощью классификатора, например AlexNet \cite{lib-alexnet}. Такой подход является довольно точным, но не эффективен как по скорости, так и по памяти. Ярким примером такого подхода могут послужить такие архитектуры как: R-CNN и Faster R-CNN \cite{lib-rcnn}.

One-stage -- более быстрый и состоит из одного шага. Сначала изображение попадает в backbone -- базовую сверточную нейронную сеть. Сигналы из разных ее уровней пробрасываются в сеть трансформирующую признаки, а затем, производится классификация и детекция. Представителями этого подхода являются: SSD \cite{lib-ssd}, YOLO \cite{lib-yolo}, RetinaNet \cite{lib-retinanet}, TTFNet \cite{lib-ttfnet}.

\subsection{Выбор подходящего детектора}

В качестве кондидатов мною были выбраны лучшие архитектуры за последние 7 лет. Отбор производился на основе соремнований для задачи детекции MS COCO. Ниже приведен список этих типов СНС:
\begin{itemize}
    \item MobileNet-v3 + SSD -- one-stage детектор симейства SSD с MobileNet (v3) в качестве backbone;
    \item YOLO-v4 (Darknet-53) -- one-stage детектор симейства YOLO v4 с Darknet-53 в качестве backbone;
    \item EfficientDet-D3 -- one-stage детектор симейства Efficientdet с EfficientNet-B3 в качестве backbone;
    \item TTFNet-53 -- one-stage детектор симейства TTFNet с Darknet-53 в качестве backbone;
    \item RetinaNet (Resnet-50) -- one-stage детектор симейства RetinaNet с Resnet-50 в качестве backbone;
    \item Faster RCNN -- two-stage детектор симейства Faster RCNN.
\end{itemize}

Все перечисленные выше кондидаты были поставлены в одиноковые условия и обучались по одному алгоритму:
\begin{itemize}
    \item СНС инициализировалась со случайными весами;
    \item СНС обучалась на выборке MS COCO (100 эпох);
    \item Обученная на предыдущем шаге СНС дополнительно обучалась на LaDD и VisDrone (10 эпох).
\end{itemize}

Результаты эксперемента приведены в таблице \ref{leaderboard-table}.

\begin{table}[H]
    \caption{Сравнение различных архитектур детекторов}\label{leaderboard-table}
    \begin{tabular}{|p{4cm}|p{3cm}|p{3cm}|p{5cm}|}
    \hline
    {Тип} & {mAP (LaDD)} & {mAP (VisDrone)} & {Время обработки изображения (Tesla v100)} \\
    \hline
    MobileNet-v3 + SSD & 0.46 & 0.12 & 100 мс \\
    \hline
    YOLO-v4 (DarkNet-53) & 0.52 & 0.15 & 270 мс \\
    \hline
    EfficientDet-D3 & 0.66 & 0.23 & 400 мс \\
    \hline
    TTFNet-53 & 0.65 & 0.21 & 300 мс \\
    \hline
    RetinaNet (ResNet-50) & 0.71 & 0.25 & 300 мс \\
    \hline 
    Faster RCNN & 0.72 & 0.24 & 500 мс \\
    \hline
    \end{tabular}
  \end{table}

  Из эксперимента видно что по совокупности точности и скорости работы лучше всего справляется с детекцией RetinaNet. Также превосходтсво этой архитектуры подтверждает еще одно исследование. В нем сравнения СНС проводились на выборке Stenfird Drone Dataset (SDD) в которой содержится около 1 миллиона изображений снятых с БПЛА в кампусе Стенфордского Университета. Некоторые результаты этого исследования приведены в таблице \ref{leaderboard-table-sdd}.

  \begin{table}[H]
    \caption{Сравнение различных архитектур детекторов на выборке SDD}\label{leaderboard-table-sdd}
    \begin{tabular}{|p{7cm}|p{5cm}|}
    \hline
    {Тип} & {mAP (SDD)} \\
    \hline
    SSD (ResNet-50) & 0.80 \\
    \hline
    Faster RCNN (ResNet-50) & 0.83 \\
    \hline
    RetinaNet (Resnet-50) & 0.85 \\
    \hline
    \end{tabular}
  \end{table}

  Из вышеперечисленных исследований следует, что RetinaNet в сравнении с другими архитектурами обладает высокой точностью и скоростью работы и ее можно применять для решения задач детектирования объектов по снимкам с БПЛА.


\clearpage