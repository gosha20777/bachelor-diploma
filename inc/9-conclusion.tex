\anonsection{Заключение}

В данной работе рассматривалась и решалась задача детектирования потерявшихся людей на аэрофотоснимках с БПЛА с помощью моделей машинного глубокого обучения. В ходе исследования были получены следующие результаты:
\begin{itemize}
    \item Проведен анализ поисково-спасательных операций, на основе которого, выявлены критерии съемки изображений и условия использования разрабатываемого алгоритма;
    \item Произведен сбор и подготовка обучающей выборки, необходимой для обучения нейросетевых математических моделей;
    \item Сделан сравнительный анализ различных архитектур СНС для решения поставленной задачи, в результате которого, была выбрана архитектура RetinaNet-ResNet50;
    \item Проведен разбор и анализ выбранной архитектуры;
    \item Проведено ряд исследований и экспериментов, направленных на повышения точности модели, в результате чего, была разработана новая архитектура СНС RetinaNet-ResNet50-WFPN (взвешенная FPN) достигшая точности детектирования mAP 93\%;
    \item Произведен ряд оптимизационных работ, направленных на уменьшение времени работы СНС на ПЭВМ, в результате чего, была достигнута скорость обработки изображения в 800 мс на центральном процессоре и 300 мс на графическом ускорителе;
    \item Разработано пользовательское программное обеспечение с графическим интерфейсом, способное работать на современных операционных системах Linux, Windows, OSX;
    \item Разработанное ПО было внедрено в поисково-спасательные отряды, собрана статистика его использования. 
\end{itemize}

Полученная модель глубокого бучения обладает довольно высоким показатель mAP как для решения поставленной задачи, так и для других задач детектирования мелких объектов. Архитектура RetinaNet-ResNet50-WFPN, также, является уникальной и содержит в себе научную новизну. Благодаря разработанному пользовательскому ПО, а также, применению различных оптимизаций, ускоряющих алгоритм, данную технологию можно успешно применять в поисково-спасательных операциях, снижая время поиска пропавшего человека. Это придает проекту высокую социальную значимость, ведь он может спасти не одну человеческую жизнь.

\clearpage