\anonsection{Заключение}

В данной работе рассматривалась и решалась задача детектирования потерявшихся людей на аэрофотоснимках с БПЛА с помошью моделей машинного глубокого обучения. В ходе исследования были получены следующие результаты:
\begin{itemize}
    \item проведен анализ поисково-спасательных операций на основе которого выявлины критерии съемки изображений и условия использования разрабатываемого алгоритма;
    \item произведен сбор и подкотовка обучающей выборки, необходимой для обучения нейросетевых математических моделей;
    \item сделан сравнительный анализ различных архитектур СНС для решения поставленой задачи в результате которого была выбрана архитектура RetinaNet-ResNet50;
    \item проведен разбор и анализ выбранной архитектуры;
    \item проведено ряд исследований и эксперементов направленных на повышения точности модели а результате чего была разработана новая архотектура СНС RetinaNet-ResNet50-WFPN (взвешенная FPN) достигшая точности детектирования mAP 93\%;
    \item ряд оптимизационных работ направленых на уменьшение времени работы СНС на ПЭВИ в результате чего была достигнута скорость обработки изображения в 800 мс на центральном процессоре и 300 мс на графическом ускорителе;
    \item разработанно пользовательское программное обеспечение с графическим интерфейсом способное работать на современных операционных системах Linux, Windows, OSX;
    \item разработанное ПО было внедрено в поисково-спасательные отряды, собрана статистика его использования. 
\end{itemize}

Полученая модель глубокого бучения обладает довольно высоким показатель mAP как для решения поставленой задачи, так и для других задач детектирования мелких объектов. Архитектура RetinaNet-ResNet50-WFPN также яаляется уникальной и содержит в себе научную новизну. Благодаря разработанному пользовательскому ПО а также применению различных оптимизаций, ускоряющих алгоритм, данную технологию можно успешно применять в поисково-спасательных операциях, снижая время поиска пропавшего человека. Это придает проекту высокую социальную значимость, ведь он может спасти не одну человеческую жизнь.

\clearpage