\subsection{Подсети классификации и регрессии}

Classification и regression subnet получают на вход признаки из низходящей пирамиды FPN и выдают предсказания о классе и местоположении объекта. 

В подсети классификации сигнал сперва проходит через 4 сверточных слоя с размерностью $[3 \times 3]$ и 256 филътрами, со стоящей после ReLU активацией. Таким образом, в каждом comvolution слое фармируется тензор размера $[h \times W \times 256]$ (256 карт призноков). Затем сиuнал снова проходит через сверточный слой $[3 \times 3]$ но уже с $K \times A$ фильтрами и сигмоидальной активацией. В результате на выходе этой сети формируется вектор длиной $K \times A$, где $K$ -- количество разных классов (в нашем случае только один класс -- это Pedestrian), $A$ -- колличество анкоров. Подсеть для каждого анкора выдает one-hot вектор где позиция числом 1 соответствует номеру класса для каждого анкора к которому модель отнесла обьект.

В подсети регрессии первые 4 слоя эквиволентны соответствующим слоям в classification subnet. Затем идет свертка  $[3 \times 3]$ формирующая $4 \times A$ карт признаков. После чего аналогично классификационной подсети формируется вектор длинны $4 \times A$. Таким образом подсеть позволяет уточнить 4-компонентный вектор координат анкора под реальный размер объекта: $(\Delta x_{min}, \Delta y_{min}, \Delta x_{max}, \Delta y_{max})$.