\subsubsection{Функция потерь}

Отметим, что функция потерь (loss function) складывается из двух компонент: ошибки регрессии и классификации (формула \ref{eq-1}).

\begin{equation}\label{eq-1}
    L = \lambda L_{reg}+L_{cls}
\end{equation}
где:
\begin{itemize}
    \item $L$ -- функция потерь RetinaNet;
    \item $L_{reg}$ -- ошибки регрессии;
    \item $L_{cls}$ -- ошибки классификации;
    \item $\lambda$ -- коэффициент усиления, определяющий соотношение между потерями.
\end{itemize}

Рассмотрим из чего складываются потери регрессии. Мы знаем, что каждому объекту ставится в соответствии анкерная рамка. Пусть $G$ -- целевой (ground truth) объект, а $A$ -- анкер. Тогда, сопоставив объекты и anchor box-ы получим $s$ пар: $(G_i, A_i), i=1..s$.

Заметим, что для каждого анкера regression subnet отдает вектор, показывающий разницу между границами якорной рамки и объекта: $(\delta x_{min}, \delta y_{min}, \delta x_{max}, \delta y_{max})$.  Подсчитав действительное расстояние между анкером и границами объекта  $(\Delta x_{min}, \Delta y_{min}, \Delta x_{max}, \Delta y_{max})$ и сравнив его с результатами работы подсети, вычислим потери регрессии:
$$
L_{reg} = \sum_{i, j} smooth_{L1}(\delta_{ij}-\Delta_{ij})
$$
где:
\begin{itemize}
    \item $\delta_{ij}$ -- прогнозируемое расстояние между anchor box-ом и границами объекта;
    \item $\Delta_{ij}$ -- реальное расстояние между anchor box-ом и границами объекта;
    \item $i \in {x, y},\ j \in {min, max}$;
    \item $smooth_{L1}(x) = \begin{cases}\frac{1}{2}x^2, & \mid x\mid < 0\\\mid x\mid - \frac{1}{2}, & x \geq 0\end{cases}$
\end{itemize}

Для потерь классификации используется функция Focal Loss \cite{lib-focal-loss}:
$$
L_{cls} = -\sum_{i=1}^K \alpha_i \cdot log(p_i)(1-p_i)^\gamma
$$
где:
\begin{itemize}
    \item $K$ -- количество классов;
    \item $p_i$ -- вероятность с которой класса $i$ был предсказан;
    \item $\gamma$ -- фокальный коэффициент;
    \item $\alpha_i$ -- коэффициент усиления класса $i$ (в нашем случае мы имеем только один класс и $\alpha$ = 1).
\end{itemize}

Данная функция есть усовершенствование функцией кросс-энтропии \cite{lib-focal-loss}, где от модели требуется высокая степень "уверенности" и влияние часто встречающихся классов возрастает. В Focal Loss это влияние, наоборот, снижается, а наибольший вклад при обучении весов RetinaNet оказывают редко встречающиеся объекты. Делается это за счёт множителя $(1-p_i)^\gamma$, а также параметров $\alpha_1$ и $\gamma \in(0, \infty)$. Графики функций focal и cross entropy представлены на рисунке ниже:

\addimghere{5-5-focal-loss}{0.8}{Графики focal loss и cross entropy}{focal-loss}

Стоит отметить, что во время обучения, большая часть объектов, обрабатываемых классификатором, является фоном, который является отдельным классом. Поэтому может возникнуть проблема, когда нейросеть обучится определять фон лучше, чем другие объекты. Такая несбалансированность очень характерно для решаемой задачи детектирования пропавших людей. По этому, в качестве функции потерь была выбрана Focal Loss.