\subsection{Предобучение на больших данных} \label{sect-6-2}

Важную часть в области глубокого обучения играют данные. Как правило, с ростом числа обучающей выборки растет и обобщающая способность нейронных сетей, а значит, и точность обработки изображений. За последние несколько лет наибольшего успеха в обучении СНС на больших объемах данных добилась команда исследователей из Facebook Research. В своем эксперименте \cite{lib-insta-net} ученые изучили насколько увеличение количества данных влияет на точность распознавания. Исследователи обучили сверточную нейронную сеть ResNet на 3,5 миллиардах изображений из социальной сети Instagram и сравнили ее с моделью обученной на ImageNet (1,2 миллиона изображений). Результат оказался впечатляющим -- полученная модель достигла качества 81.2\% accuracy на ImageNet, что является лучшем показателем в данном соревновании. 

\addimghere{6-2-instagram-resnet}{0.6}{Сравнение accuracy моделей с предобучением на 3,5 млрд. изображений и без}{imagenet-stat}

В связи с этим, я решил предобучить RetinaNet на большем по сравнению с MS COCO наборе данных и посмотреть на результаты. В качестве обучающей выборки был выбран OpenImagesDataset (OID) -- наибольший из доступных наборов данных для задачи детекции \cite{lib-iod}. Ниже приведено сравнение наборов данных OID и MS COCO:

\begin{table}[H]
    \caption{Характеристики наборов данных}\label{datasets}
    \begin{tabular}{|c|c|c|c|}
        \hline
        {Тип} & {Размер обучающей выборки} & {Число объектов} & {Число классов} \\
        \hline
        OID & 1.743 тыс. & 14.6 млн. & 600 \\
        \hline
        MS COCO & 200 тыс. & 1.5 млн. & 80 \\
        \hline
    \end{tabular}
\end{table}

За основу была взята модель обученная на MS COCO, полученная ранее, и обучена уже описанным выше образом: сначала СНС обучалась на OID наборе данных (100 эпох), затем лучшая модель (mAP = 0.4326) была последовательно обучена на VisDrone и LaDD наборах данных (по 10 эпох). Сравнительная таблица с результатами полученных моделей приведена ниже.

\begin{table}[H]
    \caption{Сравнение моделей предобученных на MS COCO и OID}\label{leaderboard-2}
    \begin{tabular}{|p{7cm}|p{5cm}|}
        \hline
        {Выборка} & {mAP} \\
        \hline
        \multicolumn{2}{|c|}{С предобучением на MS COCO} \\
        \hline
        VisDrone & 0.25 \\
        \hline
        LaDD & 0.82 \\
        \hline
        \multicolumn{2}{|c|}{С предобучением на IOD} \\
        \hline
        VisDrone & 0.34 \\
        \hline
        LaDD & 0.87 \\
        \hline
    \end{tabular}
\end{table}

Как и ожидалось, увеличение количества данных привело к улучшению обобщающей способности СНС и уточнению скрытых представлений данных, благодаря чему, точность распознавания существенно возросла.