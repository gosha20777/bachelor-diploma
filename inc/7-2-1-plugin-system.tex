\subsubsection{Запуск моделей глубокого обучения в .NET}

Как говорилось в разделе \ref{sect-7-1} для высокопроизводительного инференса моделей глубокого обучения требуются низкоуровневые библиотеки. Эти библиотеки спроектированны для разного оборудования и имеют различные интерфейсы взаимодействия (API). Также зачастую для возможности использования того или иного оборудования необходима установка различных драйверов (например nVidia CUDA и CuDNN). В добавок использование нейронных сетей часто требует наличие интерпритатора python и различных python замисимостей (например opencv-python, numpy и т д). Все это требует от пользователя дополнительныъ знаний и затрудняет установку ПО и использование СНС.

Для решения этой проблемы была разработана система модулей. Основная идея заключается в том что модель поставляется вместе с модулем, включающем в себя весь набор необходимых низкоруровневых библиотек, необходимых для работы того или оного оборудования а также программный код оссушествляющий связку между С\slash С++ методами библиотек и .NET CLI (Common Language Infrastructure). Таким образом программный код C\# способен вызывать соответствующие низкоуровневые методы и осуществлять вычисления на том или ином устройстве.

В свою очередь каждый модуль предоставляет единообразный интрефейс взаимодействия -- IObjectDetectionPlugin. С помошью него программа может узнать на каких операционных системах способен работать модуль и с какими устройствами. Модули могут подключаться в программу независимо друг от друга. ПО анализирует платформу на которой оно запущено и предлагает установить пользователю модуль, обеспечивабщий максимальную производительности на его оборудовании. Например если у пользователя имеется GPU от nVidia -- программа предложит пользователю использовать модуль с CUDA, CuDNN и TensorRT. Все модули хранятся на удаленном сервере, а пользовательское ПО способно управлять ими по средствам менеджера модулей (система управления модулей схожа с менеджерами пакетов APT, PIP, NuGet).

Классификация платформ запуска моделей глубокого обучения представлены ниже:

\begin{itemize}
    \item Операционные системы:
    \begin{itemize}
        \item Linux
        \item Windows
        \item OSX
    \end{itemize}
    \item Вычислительные устройства:
    \begin{itemize}
        \item CPU
        \item GPU
        \begin{itemize}
            \item nVidia GPU
            \item AMD GPU
        \end{itemize}
        \item Сопроцессоры
        \begin{itemize}
            \item Google Edge TPU
            \item Intel Movidius NPU
        \end{itemize}
    \end{itemize}
\end{itemize}

Для работы с различными вычислительными устройствами используются различные наборы библиотек (так например для процессоров Intel и AMD споользуется oneDNN, для GPU от nVidia -- CUDA, CuDNN и TensorRT а для GPU от AMD -- ROCm и DirectML). Применение той или иной библиотеки модет зависеть от операционной системы. С учетом этих особенностий мною были использованы следубщие варианты конфигураций:

\addimghere{7-2-plotforms-tree}{0.8}{Дерево платформ и библиотек}{7-2-plotforms-tree}

Собранный модуль меет следующую структуру: в корне модуля лежат .DET Core библиотеки обеспечивающие вызов низкоуровневых компонентов и предоставляющие программе API для инициализации и вызова модуля, а в каталоге runtimes находятся платформозависимые низкоуровневые компоненты (рис. \ref{7-2-plugin-tree}).

\addimghere{7-2-plugin-tree}{0.8}{Cтруктура модуля LаcmusRetineNetPlugin.CPU}{7-2-plugin-tree}

Благодаря предложенному подходу пользователь может легко манипулировать модулями с моделями грубокого обучения в зависимости от его потребностей. Инкапсуляция в модулях низкоуровневых библиотек избавляет пользователя от ручной установки драйверов и компонентов, облегчая использование ПО. В добавок такая система позволяет обновлять СНС без необходимости обновления самого ПО и его рекомпиляции, что упрощает процесс разработки и поддержки.

Общая схема архитектуры ПО приведена ниже:

\addimghere{7-2-app-arch}{0.8}{J,ofz fh[LacmusApp]}{7-2-app-arch}

На последок хочется отметить что данное ПО и все его компоненты (включая СНС) имеют открытый исходный код и распространяются свободно под лицензией GNU GPL v3.
