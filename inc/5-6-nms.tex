\subsection{Пост-обработка}

Зачастую получается так, что СНС отдает на выходе несколько прогнозов указывающих на один и тот же объект. В этом случае такие предсказания следует отфильтрвать, выбрав наилучшие (рис. \ref{nms}). Однако при филтрации стоит учитывать случай, когда на изображении два разных объекта одного класса могут аходиться рядом и их ограничивающие рамки могут пересекаться. Эта задача решается на этапе пост-обработки с помощью алгоритма Non-Maximun Supression (NMS).

\addimghere{5-6-nms}{0.8}{Пример работы NMS}{nms}

На вход NMS принимает набор bounding box-ов для одного класса и порог, задающий величину максимального пересечения между ними. Ограниччивающие рамки сортируются по уверенности (accuracy) и кладутся на стек. В первом цикле со стека берется очередная гепотиза. Затем во вложенном цикле со стека берется вторая гипотиза. Если между двумя гиптизами IoU больше заданого порога то вторая гепотиза отбрасывается. Алгоритм продолжает работу до тех пор пока не переберет все пары ограничивающих рамок.

Код алгоритма может выглядить так:

\lstinputlisting[numbers=left]{inc/scripts/nms.py}
