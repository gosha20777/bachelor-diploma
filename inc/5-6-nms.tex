\subsection{Пост-обработка}

Зачастую, получается так, что СНС отдает на выходе несколько прогнозов, указывающих на один и тот же объект. В этом случае, такие предсказания следует отфильтровать, выбрав наилучшие (рис. \ref{nms}). Однако, при фильтрации стоит учитывать случай, когда на изображении два разных объекта одного класса могут находиться рядом, и их ограничивающие рамки могут пересекаться. Эта задача решается на этапе пост-обработки с помощью алгоритма Non-Maximum Suppression (NMS).

\addimghere{5-6-nms}{0.8}{Пример работы NMS}{nms}

На вход NMS принимает набор bounding box-ов для одного класса и порог, задающий величину максимального пересечения между ними. Ограничивающие рамки сортируются по уверенности (accuracy) и кладутся на стек. В первом цикле со стека берется очередная гипотеза. Затем, во вложенном цикле со стека берется вторая гипотеза. Если между двумя гипотезами IoU больше заданного порога, то вторая гипотеза отбрасывается. Алгоритм продолжает работу до тех пор, пока не переберет все пары ограничивающих рамок.

Код алгоритма может выглядеть так:

\lstinputlisting[numbers=left]{inc/scripts/nms.py}