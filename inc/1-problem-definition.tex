\section{Проблема поиска пропавших людей}\label{sect-1}

Неподготовленный человек, попавший в природную среду, может легко заблудиться и пропасть. Достаточно часто люди уходят на природу, в лес -- за грибами, с прогулочными или иными целями -- теряют ориентиры, после чего самостоятельно выбраться из природных условий становится проблематично.

Розыском таких пропавших людей занимаются как и государственные структуры (МЧС, полиция), так и волонтёры в составе Поисково-Спасательных Отрядов (ПСО). Так, например, в России наиболее известными ПСО являются "Лиза Алерт", "Экстремум", "Сова", "Запад", "Орен Спас". Подобные организации распространены и в других странах по всему миру.

Так, по приблизительным оценкам в России каждый год пропадает более ста двадцати тысяч человек, в США -- более ста тысяч, на территории стран Европейского Союза эта цифра составляет приблизительно девяноста тысяч человек.

\subsection{Основные факторы риска потерявшегося человека в природной зоне}

Положение потерявшегося человека часто усугубляется из-за физических и психологических факторов, таких как:

\begin{itemize}
    \item Обезвоживание. Отсутствие доступа к питьевой воде или ограниченный её запас резко снижает возможность человека поддерживать своё стабильное состояние на протяжении длительного времени, которое может потребоваться на проведение спасательной операции;
    \item Гипотермия (переохлаждение). Зимой и в ночное время температура окружающей среды снижается (особенно в природных условиях), а у потерявшегося человека, как правило, нет возможности согреться (если нет соответствующего снаряжения). Данная причина является одной из наиболее частых причин гибели потерявшихся людей;
    \item Травмаы. Человек, который потерялся на природе, нередко начинает паниковать, что приводит к необдуманным, импульсивным поступкам, которые нередко приводят к травмам, после получения которых шансы самостоятельно добраться до цивилизации драматически уменьшаются;
    \item Паника. Осознание потерявшимся факта невозможности самостоятельно выбраться из локации нередко приводит к панике, что резко снижает вероятность успешного выхода потерявшегося человека к цивилизации.
\end{itemize}

Эти и многие другие факторы резко снижают время выживания потерявшегося человека в лесу, что в свою очередь повышает требования к временному ресурсу для поисково-спасательных операций. В среднем потерявшейся человек может продержаться в лесу в течении 6-8 дней, летом и 2-4 дней зимой. Стоит также учесть, что поиск человека может начаться не сразу, а спустя какое-то время. Иными словами -- часто счет идет на часы.
\subsubsection{Основные методики поиска пропавших людей}

Наиболее распространенным способом поиска потерявшегося человека в природной локации является наземная поисковая операция. Это, как правило, пеший поиск с участием ответственных за подобные мероприятия служб или силами волонтёров. 

Район поиска делится на квадраты размером 500$\times$500 метров, и прочесывается отрядом спасателей по заранее оговоренному маршруту (как правило цепью), с целью постараться увидеть или получить отклик потерявшегося человека. 

Данный способ поиска является несомненно точным, но достаточно тяжелым -- необходимо большое количество подготовленных к задаче поиска людей, необходима организация поисковых групп, необходимо выполнение многих других смежных задач. Помимо прочего, пеший поиск является достаточно медленным. Поисковый квадрат размером в лесной зоне закрывается пешей поисковой группой в среднем за 3-6 часов (а таких квадратов может быть несколько).

Другим способом поиска является поиск по фотографиям, с полученных с БПЛА. В последнее время появились небольшие БПЛА и их все чаще применяют спасатели для проведения поисковых работ. 

Над районом поиска составляется маршрут, по которому затем пролетает летательный аппарат, в заранее заданных точках делает фотографии с гео-меткой. После выполнения полётного задания БПЛА возвращается в поисковый штаб, где его оператор меняет на нём аккумулятор и карту памяти со снимками, после чего, загружает следующее полётное задание и дрон отправляется на дальнейшие съёмки с воздуха.

Опытным путём было установлено, что наиболее оптимальная высота для осуществления аэрофотосъёмки региона поиска приблизительно 50 метров. Данная отметка выше крон деревьев, при этом позволяет получить достаточно чёткие фотоснимки, на которых можно заметить человека.

С одной поисковой операции в среднем набирается около четырех тысяч фотографий. Полученные фотографии анализируются специалистами из ГПА (группа просмотра и анализа) с целью найти на них человека. Человек на такой фотографии может быть частично закрыт растительностью и занимает очень мало места -- в среднем взрослый лежачий человек имеет 100 пикселей в высоту и 30 в ширину при исходном размере изображения 4000$\times$3000 пикселей. Это усложняет ручной анализ изображений и пагубно сказывается на усталости анализирующего человека. В среднем опытный специалист из ГПА тратит на анализ одной фотографии около минуты, а его концентрации внимания хватает на анализ 50-70 снимков, после чего, человеку необходим отдых. На анализ четырех тысяч фотографий (одна поисковая операция) у ГПА из 5-6 человек тратится около 6.5 часов.
\subsubsection{Постановка задачи}

Цель работы заключается в исследовании и разработке различных архитектур СНС, решающих задачу детектирования потерявшихся в лесу людей по аэрофотоснимкам полученных с БПЛА, разработке пользовательского ПО, реализующего данную технологию, с целью возможного внедрения в ПСО для непосредственного применения.

В исследовании ведется учет следующих критериев:
\begin{itemize}
    \item Точность алгоритма распознавания -- СНС должна точно находить людей, не пропускать их, а число ложных срабатываний должно быть минимальным;
    \item Скорость алгоритма распознавания -- СНС должна быстро анализировать снимки на персональных компьютерах;
    \item Потребление памяти -- количество потребляемой памяти (зависит от числа параметров СНС и количества ее слоев) должно позволять запускать СНС на персональных компьютерах.
\end{itemize}

Для реализации поставленной цели были сформулированы следующие задачи:
\begin{itemize}
    \item Сбор, анализ и подготовка исходных данных необходимых для обучения нейросетевых математических моделей;
    \item Выбор оптимальной архитектуры нейросетевой модели, проверка возможности ее улучшения, обучение модели;
    \item Проведение оптимизационных работ с целью уменьшение времени работы выбранного нейросетевого алгоритма на ЭВМ;
    \item Разработка интерфейса пользователя;
\end{itemize}

\clearpage