\subsection{Постановка задачи}

Цель работы заключается в исследовании и разаработке различных архитектур СНС решающих задачу детектирования потерявшихся в лесу людей по аэрофотоснимкам полученных с БПЛА, разработке пользовательского ПО реализующего данную технологию с целью возможного внедрения в ПСО для непосредственного применения.

В иссследование ведется учет следующих критериев:
\begin{itemize}
    \item Точность алгоритма распознования -- СНС должна точно назодить людей, не пропускать их, а число ложных сработываний должно быть минимальным;
    \item Скорость алгоритма распознования -- СНС должна быстро анализировать снимки на персональных компьютерах;
    \item Потребление памяти -- колличество потреьляемой памяти (зависит от числа гипер-параметров СНС) должно позволять запускать СНС на персональных компьютерах.
\end{itemize}

Для реализации поставленной цели были сформулированы следующие задачи:
\begin{itemize}
    \item Сбор, анализ и подготовка исходных данных необходимых для обучения нейросетевых математических моделей;
    \item Выбор оптимальной архитектуры нейросетевой модели, проверка возможносит ее улучшения и обучение модели;
    \item Проведение оптимизационных работ с целью уменьшение времени работы выбранного нейросетевого алгоритма на ЭВМ;
    \item Разработка интерфейса пользователя;
\end{itemize}