\subsubsection{Сверточный слой}

Этот слой представляет из себя набор признаков (карт), каждая из которых имеет фильтр (сканирующее ядро)\cite{lib-cnn}. Размеры всех признаков сверточного слоя -- одинаковы и определяются формулой: $(h,w) = (W_m-W_k, H_m-H_k)$. Где: $(h,w)$ -- размер карты, $W_m$ и $W_k$ -- ширина предыдущей карты и ядра, $H_m$ и $H_k$ -- их высота.

В общем виде слой слой можно описать формулой:

$$
x^i = f(x^{i-1}*k^i+b^i)
$$

Где:
\begin{itemize}
    \item $x^i$ -- выходное значение слоя $i$;
    \item $f(x)$ -- нелинейная функция активации;
    \item $k^i$ -- ядро $i$-го слоя;
    \item $b^i$ -- коэффициент сдвига слоя $i$;
    \item $*$ -- операция дискретной свертки: $(f*g)[m,n]=\sum_{k,l} f[m-k,n-l]\cdot g[k,l]$
\end{itemize}

\addimghere{2-1-2-comv-visualization}{0.5}{Операция свертки}{conv-vizialization}

Ядро представляет собой систему разделяемых весов или синапсов, это одна из главных особенностей сверточной нейросети. В многослойном перцептроне \cite{lib-perciptrone} очень много связей между нейронами, то есть синапсов, что весьма замедляет процесс детектирования. В сверточной сети -- наоборот, общие веса позволяет сократить число связей и позволить находить один и тот же признак по всей области изображения:

\addimghere{2-1-2-filter-visualization}{0.9}{Фильтр “ищет” левосторонние кривые, результат положительный}{conv-vizialization-positive}