%!TEX TS-program = xelatex

% Author: Georgy Perevozchikov (gosha20777@live.com)
% https://github.com/gosha20777/bachelor-diploma

\documentclass[a4paper,14pt]{extarticle} % 14й шрифт
\input{inc/preamble} % Подключаем преамбулу

%%% Начало документа
\begin{document}

\anonsection{Реферат}
Выпускная квалификационная работа содержит 66 страниц, 31 рисунков, 9 таблиц, 59 источников.

\MakeUppercase{сверточные нейронные сети, RetinaNet, Weight Feature Pyramid Network, WFPN, детектирование объектов на снимках, потерявшийся человек, поисково-спасательные отряды, беспилотный летательный аппарат}

Объектом исследования являются сверточные нейронные сети (СНС) для детектирования потерявшихся людей на снимках с беспилотных летательных аппаратов (БПЛА) в поисково-спасательных операциях (ПСО).

Области применения: автоматизация поиска людей в поисково-спасательных отрядах (ПСО).

Цель работы -- разработка, проектирование и программная реализация модели глубокого машинного обучения, позволяющей детектировать потерявшихся людей на аэрофотоснимках и внедрение разработанного программного обеспечения (ПО) в ПСО.

В процессе исследования проводились: обоснование актуальности разработки, обзор и сравнение различных архитектур СНС для решения поставленной задачи, ряд исследований и экспериментов направленных на повышение качества распознавания и скорости работы СНС RetinaNet.

В результате была разработана новая архитектура СНС RetinaNet с Weight Feature Pyramid Network (WFPN), позволяющая детектировать мелкие объекты (в частности, пропавших людей) точнее существующих аналогов, в чем заключается научная новизна данного исследования. Разработано пользовательское ПО, которое было успешно внедрено в различные ПСО, и собрана статистика его практического применения, что также придает данному исследованию социальное значение.

\end{document}
%%% Конец документа